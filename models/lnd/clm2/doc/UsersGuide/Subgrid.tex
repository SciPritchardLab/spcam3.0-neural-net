\section{CLM2.1 Data Structures}
\label{sec_data_structures}

In what follows, we provide a brief summary of the new CLM2.1 data
structures. Understanding of these data structures is essential before
the user attempts to modify code and/or add new history output fields
to the model.

The subgrid hierarchy in CLM2.1 is composed of gridcells, landunits,
columns and plant functional types (pfts).  Each gridcell can have a
different number of landunits, each landunit can have a different
number of columns and each column can have multiple pfts.  This
results in efficient memory allocation, and allows for the
implementation of many different types of subgrid representation.

The first subgrid level, the landunit, is intended to capture the
broadest spatial patterns of subgrid heterogeneity.  These broad
patterns include the physically distinct surface types that were
treated as special cases in the previous versions of CLM2.0
(e.g. glaciers and lakes).  In terms of CLM2.0
variables, the central distinguishing characteristic of the landunit
subgrid level is that this is where physical soil properties are
defined: texture, color, depth, pressure-volume relationships, and
thermal conductivity.  In CLM2.1, landunits are used to represent the
special landcover types (e.g. glacier and lakes), with a single
additional landunit for the gridcell vegetated area.

The second subgrid level, the column, is intended to capture potential
variability in the soil and snow state variables within a single
landunit.  The central characteristic of the column subgrid level is
that this is where the state variables for water and energy in the
soil and snow are defined, as well as the fluxes of these components
within the soil and snow.  Regardless of the number and type of pfts
occupying space on the column, the column physics operates with a
single set of upper boundary fluxes, as well as a single set of
transpiration fluxes from multiple soil levels.  These boundary fluxes
are weighted averages over all pfts.

The third and final subgrid level is referred to as the plant
functional type (pft), but it also includes the treatment for bare
ground.  It is intended to capture the biophysical and biogeochemical
differences between broad categories of plants, in terms of their
functional characteristics.  All fluxes to and from the surface 
are defined at the pft level, as are the vegetation state variables
(e.g. vegetation temperature, canopy water storage, and carbon and
nitrogen states for the leaf, stem, and roots).

In addition to state and flux variable data structures for conserved
components at each subgrid level (energy water, carbon, nitrogen,
etc.), each subgrid level also has a physical state data structure for
handling quantities that are not involved in conservation checks
(diagnostic variables).  For example, soil texture is defined through
physical state variables at the landunit level, the number of snow
layers and the roughness lengths are defined as physical state
variables at the column level, and the leaf area index and the
fraction of canopy that is wet are defined as physical state variables
at the pft level.

The hierarchical subgrid data structures are implemented in the code
through the modules {\bf clmtype.F90}, {\bf clm\_mapping.F90} and {\bf
clmpoint.F90} (all in the {\bf /src/main} subdirectory). The new code
makes extensive use of the Fortran 90 implementation of the derived
data type.  This permits the user to define new data types that can
consist of multiple standard data types (integers, doubles, strings)
as well as other derived data types.

This subgrid hierarchy is implemented in CLM2.1 as a set of nested
derived types.  The entire definition is contained in module {\bf
clmtype.F90}. Extensive use is made of pointers, both for dynamic
memory allocation and for simplification of the derived type
referencing within subroutines.  The use of pointers for dynamic
memory allocation ensures that the number of subgrid elements at each
level in the hierarchy is flexible and resolved at run time, thereby
eliminating the need to statically declare arrays of fixed dimensions
that might end up being sparsely populated.  The use of pointers for
referencing members of the derived data type within the subroutines
provides a coherent treatment of the logical relationships between
variables (e.g., the user cannot inadvertently change a pft-level
variable within a subroutine that is supposed to operate on the column
states and fluxes), and a more transparent representation of the core
algorithms (it is easy to tell when the code is in a column or pft
loop).

The module, {\bf clmtype.F90}, is organized such that derived types
which are members of other derived types are defined first (a Fortran
90 compiler requirement).  In particular, the energy and mass
conservation data types are defined first, followed by data types
constituting the pft level, column level, landunit level, gridcell
level and the model domain level. Finally, the hierarchical
organization of these types is defined, starting with the model domain
level, which consists in part of a pointer to an array of gridcells,
each of which consists in part of a pointer to an array of landunits,
each of which has a pointer to an array of columns, which each have a
pointer to an array of pfts.  The last section of {\bf clmtype.F90}
includes additional data structures that map the hierarchical
organization to simple 1d vectors at each level for history, restart
and initial file output.  

The basic functions of subroutine {\bf clm\_map()} in {\bf
clm\_mapping.F90} are memory allocation for the subgrid hierarchy and
initialization of areas and weights associated with each subgrid
component. Use is made of input gridded datasets defining the spatial
distribution of pfts and other surface types (glacier, lake, etc.).
This is the primary routine that needs to be modified in order to
accommodate different subgrid representations using the data
structures defined in {\bf clmtype.F90}.

Module {\bf clmpoint.F90} creates arrays of one-dimensional real and
integer pointers for various {\bf clmtype} derived type
components. These arrays are integral to the CLM2.1 history file
handler logic as well as the restart and initial file generation. The
exact usage of these arrays in the model's history file logic is
discussed in more detail in section \ref{sec_history_file}.

Finally, CLM2.1 includes the module, {\bf pft2columnMod.F90}, which
contains methods for averaging fluxes and states between levels in the
hierarchy (currently pft to column in the current implementation).  It
is intended that similar routines for averaging from column to
landunit and from landunit to gridcell will be added later.
