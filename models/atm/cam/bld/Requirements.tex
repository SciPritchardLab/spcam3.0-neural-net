\documentclass[]{article}
\usepackage{html}
\usepackage[T1]{fontenc}

\begin{document}
\bodytext{BGCOLOR=white LINK=#083194 VLINK=#21004A}

\begin{titlepage}

\begin{latexonly}
\noindent {\bf Community Climate System Model} \\
\noindent National Center for Atmospheric Research, Boulder, CO \\
\vspace{2in}
\end{latexonly}

\begin{center}
{\Large\bf CAM Build Script Requirements} \\
\medskip
{\it Author Erik Kluzek}
\end{center}

\end{titlepage}

\tableofcontents

\newpage

\section{Synopsis}

This document is the requirements for the configure/build/run set of scripts
to support using CAM.

\section{Description}

Requirements are enemerated with a short title and a longer description. Attributes
then describe their status (proposed, accepted, deleted, or implemented), and
testing (describing how this particular requirement can be verified).

\section{Requirements}

\subsection{System Constraints}

\begin{enumerate}

{\bf\em \item Purpose} \\
Provide a set of robust scripts to help the user create build and run any desired model
configuration on any of the supported platforms at any of the supported labs. And also
make it easy to setup at non-supported sites.

{\bf\em \item Supported Platforms}\\
Platforms that the scripts should work on are as follows:\\
\begin{itemize}
  \item IBM-SP
  \item SGI
  \item Compaq-Alpha
  \item Linux-Portland group compiler
  \item Linux-Lahey compiler
  \item Solaris
\end{itemize}
(status: accepted)

{\bf\em \item Supported Labs}\\
The scripts should work on all supported platforms at the following labs.\\
\begin{itemize}
  \item NCAR

  \item ORNL

  \item LLNL

  \item NERSC

  \item NASA-DAO
\end{itemize}

(status: accepted)

{\bf\em \item Users}\\
The system will be built with the following users in mind:
\begin{itemize}
\item University Students -- Graduate and undergraduate university students
	completely unfamiliar with CAM, and limited programming experience.
\item University Researchers -- University proffesors and scientific researchers
	familiar with older versions of the CCM and experience with FORTRAN programming.
\item Scientists -- CCSM Scientists familiar with the science in the latest version of CAM
	with extensive programming experience, but less familiar with the implimentation
	details of the current model.
\item Developers -- Software Engineers who are experts in the current version of CAM, 
       and have extensive FORTRAN and shell programming experience.

\end{itemize}
Thus the gamet of novice to expert users are targeted. Only the developer class is
expected to actually change the scripts. The scripts should be flexible and robust
enough that other users would be able to use them without changes.
\\
(status: proposed)

{\bf\em \item Assumptions}\\
The user will not be able to use these tools to create un-supported build configurations.
Warnings will be issued if the user tries to do so, and the closest supported configuration 
will be generated. Perl, standard Perl modules, GNU Make, MPI, and NetCDF libraries will 
be assumed to be loaded on the system. If possible scripts will terminate with an error
if these packages are missing.
\\
(status: proposed)

\subsection{Functional Requirements}

{\bf\em \item Separation of steps}\\
Separate the configure, build-executable, build-namelist, and run-model steps, into
either separate scripts or as different options in one or more scripts. \\
(status: implemented,
use cases: configure.pl, Makefile, build-namelist.pl, run-model.pl testing: run each 
separately)

{\bf\em \item Test scripts built on top of configure/build/run scripts}\\
Scripts for testing (both system and unit tests) should be built on top of the 
configure/build/run scripts. \\
(status: implemented, use cases: CAM\_test.pm, test-model.pl)

{\bf\em \item Interactive option prompting for input}\\
Scripts should have a mode where they prompt for needed input, as well
as a mode where everything is provided in the script. Thus users can either provide
input with command-line options, set environment variables, be prompted for input,
or hard-code settings into the script itself for the a given case.\\
(status: implemented use cases: configure.pl, run-model.pl)

{\bf\em \item Interactive and batch modes available for scripts}\\
Interactive mode as well as batch mode should be available for all tools and scripts. \\
(status: implemented use cases: run-model.pl, test-model.pl, *\_batch.csh)

{\bf\em \item Configure doesn't effect configuration files if run more than once and setup the same}\\
There should be the ability to run the configure step more than once, without impacting building 
the executable unless the configuration is changed. \\
(status: implemented, testing: run configure then build several times, ensure only builds 
when needed)

{\bf\em  \item Scripts abort on error or missing information}\\
If user doesn't provide needed information or information is incorrect scripts will terminate
or provide reasonable defaults at the earliest opportunity (library paths wrong for example).\\
(status: not completely implemented yet, testing: set various settings wrong run script and
make sure it catches the problem)

{\bf\em \item Namelist use default settings and datasets}\\
Namelist creation should include standard default settings and standard datasets when available
and applicable. Pathnames to default datasets should use the standard CCSM directory
structure.\\
(status: implemented, use cases: Default*.xml files testing: need to ensure different cases 
and model resolutions work)

{\bf\em \item Error checking on namelists}\\
If datasets in namelist aren't provided or there are obvious problems the script should terminate
with an error. Duplicated namelist-items (including different case) should be caught. Any required
input datasets should be ensured they are on the namelist and queried for existance. Required
namelist items should be checked or provided.
\\
(status: implemented, testing: run build-namelist with common known problems)

{\bf\em \item Option to copy configuration and input files to script location}\\
Since the model is often built and run on a disk that is scrubbed, there should be an option
to copy (or soft-link) configuration and input files to the directory scripts are called from.\\
(status: not implemented, testing: run with and without option, ensure files are correctly
dealt with)

{\bf\em \item Must be able to simultaneously build and run on different platforms in the same directory}\\
Using the same NFS mounted source directory the user must be able to run the same scripts on 
different platforms at the same time.
\\
(status: implemented, testing: run test-model.pl or run-model.pl on different platforms at same time) 

{\bf\em \item Provide a directory with modified code}\\
There should be a way of providing a directory where modified code resides to use
in place of standard code.\\
(status: implemented, use cases: MODEL\_MODDIR environment variable, testing: use MODEL\_MODDIR
to ensure uses right code)

{\bf\em \item Dependencies are generated from the Makefile automatically}\\
The Makefile should automatically figure out the dependencies between source files
and update the list of dependencies automatically.
\\
(status: implemented, testing: make realclean, make)

{\bf\em \item Platform is automatically detected by configure, make, and run process}\\
The configure, Makefile, and run steps should automatically detect the platform running
on and setup for running on that given platform.
\\
(status: implemented, testing: configure.pl, make, and run-model.pl on different platforms)

{\bf\em \item Scripts don't modify the users environment}\\
Scripts should not set environment variables that change the users environment after they
have been run.
\\
(status: proposed, testing: ensure scripts don't set any environment variables after running)

{\bf\em \item Support simple run-script that can be easily used on any platform at any site}\\
There should be a simple script to build and run the model that new users would be able to
easily use that can be used to easily run any model configuration on any platform at all
supported sites, and can be easily adapted to run at other sites.
\\
(status: almost implimented (need to improve run-model.pl), testing: ensure run-model.pl 
continues to work everywhere)

{\bf\em \item Support script resubmission}\\
Run scripts need the ability to resubmit themselves after they complete, provided the first
submission was successful. Implied with this is a mechanism to automatically stop resubmission 
when the simulation has proceeded far enough.
\\
(status: implimented, testing: ensure run-model.pl resubmission feature works)

{\bf\em \item Archive log-file information}\\
Run-scripts should be able to archive the log files to MSS easily with some kind
of identifiable naming convention. Also log-files should contain a copy of the namelist
and the latest description of the ChangeLog as a way to identify which model version
was used for this simulation.
\\
(status: implimented, testing: ensure run-model.pl archives logs)

{\bf\em \item For the LLNL "Compass" machines, must be able to detect if dmpirun or prun is used
to submit.}\\
Some of the LLNL machines Compass use "prun" and some use "dmpirun". Detecting which is used
needs to be detected at run-time, since when submitting to Compass you can't determine
beforehand which machine you will end up running on.
\\
(status: implimented, testing: see if it runs on Compass)

{\bf\em \item Library settings in some cases come from system environment variable settings.}\\
Some of the LLNL machines Compass use "prun" and some use "dmpirun". Detecting which is used
needs to be detected at run-time, since when submitting to Compass you can't determine
beforehand which machine you will end up running on.
\\
(status: implimented, testing: Testing on these platforms ()

{\bf\em \item Detect system environment variables for some systems} \\
Some systems (such as NERSC IBM, or NCAR SGI) use the "modules" package to set environment
variables to determine the location of standard library packages. The scripts need
to detect these environment variables and use them for the relevent locations.
\\
(status: implimented, testing: see if it runs on these machines)

{\bf\em \item Ability to store log-files in directory seperate from build/run location} \\
Some machines require that you run on directories that are immediatly scrubbed after
a batch submission. In other cases the model is run on directories that may be scrubbed
but the user may want to archive these files on a more permenent disk.
\\
(status: implimented, testing: test to see if you can change LOG\_DIR)

{\bf\em \item Use consistent interface to shared-memory and distributed-memory variables to
determine behavior} \\
Since shared-memory and distributed-memory environment variables (setting number of
threads, number of tasks per node, number of tasks on a node etcetera) are different on
different platforms the scripts should provide a way to set these variables in a 
consistent manner across all platforms. This way the user doesn't have to know the specifics
about how this is implimented on a particular platform, but can use one mechanism to control
these settings for all platforms.
\\
(status: implimented, testing: test to see if you can them)

{\bf\em \item Scripts should have the ability to clean out the old build directory} \\
The scripts should have a simple option to ensure a clean build by deleting the previous
object and module files. A similar option needs to be maintained for run-time files
as well.
\\
(status: implimented, testing: test clean-option)

{\bf\em \item Have the build-step in the scripts produce a log file} \\
When the model executable is built, create a log-file with a date in it, so
that you can examine it to understand the details of what might have gone wrong.
\\
(status: implimented, testing: look at compile\_log.atm)

\subsection{Non-Functional Requirements}

{\bf\em \item Scripts use environment variables, command-line options, Configuration
files, and default configuration files to determine settings}\\
Scripts should allow the use of a limited set of environment variables
to determine some settings. Environment variables should only be allowed for system
dependencies such as compilers, compiler-options, location of GNU-Make, etcetera. 
Command-line options should also be allowed in scripts to override
default settings, as well as directly editing configuration files. All default settings
should all be set in a set of default configuration files. 
(status: implemented, use cases: configure.pl, Makefile, build-namelist.pl, run-model.pl 
testing: run scripts both ways)

{\bf\em \item User should not have to edit scripts at all}\\
The user should not have to edit any scripts in order to build or run particular configurations.
The user-interface to the scripts should allow the ability of the user to specify
the specifics of the model configuration they wish to run based on the user-interface 
outlined in requirement 2.
(status: implemented, use cases: run-model.pl, test-model.pl)

{\bf\em \item Modular and extensible}\\
Users should be able to take modular components of the scripts to create their own extensions
or modifications to the existing build toolkit. \\
(status: implemented, use cases: run-model.pl, vs test-model.pl, cam-run.pl)

\subsection{Project Issues}

{\bf\em \item Standardized data formats used}\\
If datasets are used, standard data formats such as XML should be used. And parsers for
these formats can be found from existing open-source or other free software.
\\
(status: implemented, use cases: atmlndnl.pm uses XML files and the CPAN XML::Lite parser)

{\bf\em \item User Documentation}\\
User documentation created from scripts using Perl-doc. This allows the automatic
creation of documentation in both man-page and HTML formats.
(status: )

{\bf\em \item User-interface for all scripts consistent across all platforms}\\
User-interface for scripts on all supported platforms is the same.\\
(status: implemented, use cases: run-model.pl)

{\bf\em \item Defaults for supported labs are easily provided}\\
Defaults for supported labs will be provided easily. \\
(status: implemented, use cases: run-model.pl, test-model.pl, CAM\_lab.pm)

{\bf\em \item Needed settings for non-supported labs are easily provided}\\
Settings needed for non-supported labs can be easily provided. \\
(status: implemented, testing: run with LAB=default, make sure can provide needed settings 
by hand)

{\bf\em \item Adding new platforms is easy}\\
Adding a new platform in the scripts should be straight-forward. \\
(status: implemented , use cases: CAM.pm, CAM\_run.pm, CAM\_lab.pm see
section "How to add another platform" below on how to do this) 


\end{enumerate}

\section{Glossary}

\begin{tabular}{r p{3.3in} p{4in}}

{\bf autoconf:} & A GNU utility program that helps to maintain source code build configurations
for different platforms. \\ \\

{\bf configure:} & Create the files needed in order to build the model executable. This
includes determining which code needs to be compiled and resolution. It ussually involves
creating the CPP files and Makefile. \\ \\

{\bf CPAN:} & Comprehensive Perl Archive Network. A web-accessible site that contains
a host of code written in Perl by others to do various things. Since, people around the
world use and contribute to software here, it tends to be well-tested and of high-quality.
If you can find something to use from here it's likely that you can save time from having
to write the code yourself. \\ \\

{\bf CPP:} & "C" Pre-processor. A standard syntax for "editing" source code
by replacing special "CPP" syntax with code that can be compiled.  \\ \\

{\bf C-shell:} & A particular UNIX shell that is often used for creating utility scripts. \\ \\

{\bf distributed-memory:} & Parallel processing on machines where the memory is not
shared between processors. This is done by passing data between the different CPU's. \\ \\

{\bf Environment variables:} & Variables that are set in the UNIX shell that retain
their value in sub-processes. \\ \\

{\bf GNU:} & A set of UNIX shell programs and utilities created using a Open-source
code paradigm by the Free Software Foundation. (The acyonym stands for GNU's NOT UNIX). \\ \\

{\bf Makefile:} & A utility to maintain a consistent build mechanism for program
source codes. \\ \\

{\bf Object Oriented and Objects:} & A programming methodology, where data and methods
are tied together. An object creates a set of data that it will use object methods to
operate on. The data in the object is private to the object. \\ \\

{\bf Perl:} & A scripting language (Practical Extraction and Report Language). \\ \\

{\bf soft-link:} & A UNIX command that provides a "hook" to a file in another location.
It creates a new name that allows you to access a file in another location. \\ \\

{\bf SPMD:} & Single Program Multiple Data. A method of distributed memory programming
where a single program is used, but different machine CPU's operate on different parts
of the data. \\ \\

{\bf XML:} & eXtensible Markup Language. A industry standard language that allows the
creation of simple text databases. \\ \\

\end{tabular}

\section{Review Status}

\noindent {\bf Requirements Review} \\

\begin{tabular}{r p{1.3in} p{2in}}
{\bf Review Date:} & Feb/22/2002 \\ \\
{\bf Reviewers:}   & Brian Eaton         & NCAR \\
                   & Mariana Vertenstein & NCAR
\end{tabular}

\end{document}
