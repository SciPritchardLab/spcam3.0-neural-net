\documentclass[12pt]{article}
\setlength{\oddsidemargin}{0in}
\setlength{\evensidemargin}{0in}
\setlength{\textwidth}{6.5in}
\setlength{\textheight}{9in}
\setlength{\topmargin}{0in}
\setlength{\headheight}{0in}
\setlength{\headsep}{0in}
\setlength{\topskip}{0in}
\newlength{\parindentorig}
\setlength{\parindentorig}{\parindent}

\usepackage{html}

\newcommand{\code}[1]{\texttt{#1}}

\newcommand{\prototype}[1]{\noindent\texttt{\textbf{#1}}}



\begin{document}
\bodytext{BGCOLOR=white LINK=#083194 VLINK=#21004A}
\pagenumbering{roman}

%================================================================================
\begin{titlepage}

\begin{latexonly}
\noindent {\bf Community Land Model} \\
\noindent National Center for Atmospheric Research, Boulder, CO \\
\vspace{2in}
\end{latexonly}

\begin{center}
{\Large\bf Time Manager Module: Requirements and Design} \\

\bigskip
{\it B. Eaton and M. Vertenstein }

\bigskip
20 November 2001
\end{center}

\begin{htmlonly}
\begin{flushleft}
Printed versions available:
\htmladdnormallink{ps}{http://www.ccsm.ucar.edu/models/lnd-clm2/docs/time-manager.ps},
\htmladdnormallink{pdf}{http://www.ccsm.ucar.edu/models/lnd-clm2/docs/time-manager.pdf}
\end{flushleft}
\end{htmlonly}

\begin{latexonly}
\vspace{2in}
\begin{flushleft}
Online version:\\
http://www.ccsm.ucar.edu/models/lnd-clm2/docs/time-manager/index.html
\end{flushleft}
\end{latexonly}

\end{titlepage}

%================================================================================
\tableofcontents
\newpage

%================================================================================
\section{Overview}
\pagenumbering{arabic}

This document describes the requirements and design of the time manager
module in CLM2.  The time manager is built
on top of the utilities for time management available from the Earth System
Modeling Framework (ESMF) \cite{esmf} libraries.

The time manager's user interface is provided via namelist variables
which may be set to control simulation properties such as the timestep
size, start date, and stop date.  Dates are specified using either a
Gregorian or a 365-day calendar.

The time manager's API provides the methods that are used to control the
model's time loop, and provides date and time information to model
procedures or modules that may need it.  The API also provides an alarm
facility which is designed to be used by parameterizations that need to
carry out certain actions at specified times during the simulation.  The
alarms are set up during the initialization process, and are queried each
timestep for their on/off status.

%================================================================================
\section{Terminology}

As a model advances in discrete timesteps the time manager keeps track of
the date and time at each endpoint of the current timestep.  We begin with
this section on terminology to clarify our use of common words like
``date'' and ``time.''

\begin{description}

\item [date] \label{term:date} 
The term ``date'' is used to refer to an instant in time.  It consists of
year, month, day of month, and time of day components.  The time of day is
expressed in UTC.  A date specification is incomplete without it's
associated calendar.

\item [time] \label{term:time}
The term ``time'' is used in the sense of ``simulation time'' and expresses
an elapsed time since a reference date.  These time values follow the
convention for time coordinates supported by the COARDS \cite{coards} and
CF \cite{cf} metadata conventions.

\item [time of day] \label{term:tod}
Time of day refers to the elapsed time since midnight of the current day.
We express time of day in UTC unless explicitly stated otherwise.  In
common usage the term ``time'' refers to ``time of day.''  In this document
we will use ``time of day'' explicitly when that is what we mean.

\item [start date] \label{term:start date}
The start date of a simulation is the date assigned to the initial conditions.

\item [reference date] \label{term:reference date}
The reference date of a simulation is an arbitrary date that corresponds to
the origin of the time coordinate.  Often this date corresponds to the
start date.  But it may be different because providing a reference date
that is common to a set of simulations that may have different start dates
allows them to use the same time coordinate.

\item [current date] \label{term:current date}
A simulation advances by timesteps.  At any point during a simulation
the current date is taken to be the date at the end of the current
timestep.

\item [start time] \label{term:start time}
The start time of a simulation is the elapsed time from the reference date to
the start date.

\item [current time] \label{term:current time}
The current time of a simulation is the elapsed time from the reference date to
the current date.

\item [calendar day] \label{term:calendar-day}
The day number in the calendar year.  January 1 is calendar day 1.
Calendar day may be expressed in a floating point format consisting of the
integer day number plus the time of day (UTC) represented as a fractional
day.  For example assuming a Gregorian calendar:

\begin{tabular}{ll}
Date & Calendar day \\
\hline
10 January 2000, 6Z    &  10.25   \\
31 December 2000, 18Z  &  366.75
\end{tabular}

\item [restart date] \label{term:restart-date}
The restart date of a simulation is the date of the data in the restart
files from which the simulation is continuing.

\end{description}

%================================================================================
\section{Requirements}

%--------------------------------------------------------------------------------
\subsection{Date calculations}

\begin{itemize}

\item
Support Gregorian and ``no-leap'' calendars.  A ``no-leap'' calendar is the
same as a Gregorian calendar with no leap years.

\item
Represent dates and times to a precision of 1 second.

\item
Date and time representations must have a range of at least 200,000 years.

\end{itemize}

%--------------------------------------------------------------------------------
\subsection{Simulation control}

\begin{itemize}

\item
The time manager will be initialized by specifying the calendar type,
timestep size, start date and stop date.  The user may optionally specify
the reference date.  By default the reference date equals the start date.

\item
The timestep size must be chosen so that there are an integral number of
steps in a day.  

\item
The stop date may optionally be specified by any of the following:
\begin{itemize}
\item a number of timesteps from the start date;
\item a number of days from the start date;
\item a number of timesteps from the restart date;
\item a number of days from the restart date;
\end{itemize}

\item
The current date may be changed at any point during a simulation.

\item
Provide methods to query timestep size, start date, stop date, and
reference date.

\item
Provide methods to query the following:
\begin{itemize}
\item the start date, stop date, and reference date;
\item current timestep number and size;
\item current date, time, and calendar day;
\item previous date, i.e., date at the beginning of the current timestep;
\item perpetual date,
\item if the current timestep is the final one of the day or month;
\item if the current timestep is the first one of an initial or restart run;
\item if the current timestep is the last one of the run;
\end{itemize}

\item
The methods that return the current date and calendar day should provide
for an optional offset to be specified.  This provides easy access to the
calendar calculations required to find date or calendar day values that
are offset from the end-point of the current timestep.

\item
Provide an option to allow running the simulation in a ``perpetual
calendar'' mode.  Under this option the time and dates are always available
both as the usual simulation time and date, and as a perpetual time and
date.  The simulation time and date are written to the output history files
and log files to track the simulation progress with a monotonic time
progression.  The perpetual time and date are used to determine the sun
position and interpolate boundary datasets.  The perpetual day (i.e., the
year, month, and day of month part of the date) may either be read from the
initial file or may be set to user specified values.  When running in
aqua-planet mode the perpetual day is set to March 21 of year 0.  The
perpetual date and calendar day include a diurnal cycle.
\begin{itemize}
\item
A query method will be provided to determine if the time manager is using
the perpetual calendar mode.
\item
A query method will be provided that returns the current perpetual date
which is the perpetual year, month, and day of month, plus the time of day
component from the simulation's current date.  This forces the time of
day written to the history files to be consistent with the phase of the
diurnal cycle that is used in the solar calculations.
\item
The method that returns the current calendar day will return the calendar
day corresponding to the current perpetual date when running in perpetual
mode.
\end{itemize}

\item
The time manager must be able to "restart", i.e., it must be able to write
its state to a binary restart file, and reset its state after reading the
restart file.  The only attribute of the time manager state that may be
changed when initializing a restart run is the stop date.

\end{itemize}

%--------------------------------------------------------------------------------
\subsection{Alarms}

An alarm is used to signal that some event should take place during the
current timestep.  Alarms are initialized by specifying a set of times when
they should be turned on.  The time manager will turn an alarm on during
the first timestep whose late end-point (i.e., the current time) equals or
exceeds the current alarm time.  When an alarm is turned on the alarm time
is updated.  Some specific requirements for alarm functionality are:

\begin{itemize}
\item
Provide alarms that are turned on periodically and can be specified by a period
and an offset.

\item
Provide alarms that are turned on when year and month boundaries are crossed.

\item
Provide method to query whether an alarm is on or off.

\item
Provide methods to query the current alarm time and date.  The current
alarm time is the time that the alarm is set to be turned on.  If
the alarm in currently on, then the current alarm time is the next time
that the alarm will be turned on since the time manager updates the alarm
time when it turns an alarm on.

\end{itemize}

%================================================================================
\section{Interface Design}

The time manager is designed as a module which provides public data members
for communicating with the user interface (i.e., Fortran namelist), and
public methods which wrap the ESMF library methods.  The reasons for
designing ``wrapper'' methods rather than making direct use of the methods
provided by the ESMF library's Fortran 90 interface are:

\begin{itemize}
\item to provide a simpler, higher level CLM2 specific interface;
\item to hide the handling of the ESMF error returns;
\item to hide the interprocess communication necessary when running with MPI;
\item to provide methods for CLM2 specific file I/O to support restarts;
\item to provide CLM2 specific simulation control options.
\end{itemize}

The ESMF library has an object oriented design.  In Fortran this implies a
viewpoint in which each variable declared as one of the derived types is
thought of as an independent object.  Thus, for example, different objects
(variables) of the ESMF defined type for representing a date could be based
on different calendars, and there could be multiple instances of the time
manager.  But in the context of a single land model simulation there is
only one calendar being used, and only one time coordinate.  Hence, only
one time manager is necessary.  The CLM2 specific interface is then
simplified, for example, by maintaining the calendar type and the time
manager ID as private module data which does not have to be passed through
the public method argument lists.

The ESMF library methods all provide an optional argument for error
checking.  The CLM2 specific interface does not pass this argument, but the
methods act as wrappers that carry out checking of all error codes and
invoke a CLM2 (or CCSM) specific error handler as appropriate.

There is no CLM2 specific simulation control option currently implemented.

%--------------------------------------------------------------------------------
\subsection{User interface}

The user interface is implemented via Fortran namelist variables.  These
variables are public data in the time manager module.

%................................................................................
\subsubsection{\code{calendar}}

\begin{tabular}{lp{5.5in}}
Type    & character(len=*) \\
Default & 'NO\_LEAP'  \\
Use     &  
Calendar to use in date calculations.  Supported calendars are 'NO\_LEAP'
(i.e., modern calendar, but without leap years) and 'GREGORIAN' (modern
calendar).
\end{tabular}

%................................................................................
\subsubsection{\code{dtime}}

\begin{tabular}{lp{5.5in}}
Type    & integer \\
Default & 1200 s \\
Use     &  
Timestep size in seconds.  \code{dtime} must evenly divide the number of
seconds in a day.
\end{tabular}

%................................................................................
\subsubsection{\code{start\_ymd}}

\begin{tabular}{lp{5.5in}}
Type    & integer \\
Default & Read from the initial conditions input dataset \\
Use     &  
Year, month and day of the simulation start date.  The values are encoded
in an integer as \code{year*10000 + month*100 + day}.
\end{tabular}

%................................................................................
\subsubsection{\code{start\_tod}}

\begin{tabular}{lp{5.5in}}
Type    & integer \\
Default & 0 if \code{start\_ymd} is specified; otherwise it's read from the
initial conditions input dataset \\
Use     &  
Time of day (UTC) of the simulation start date, in seconds.
\end{tabular}

%................................................................................
\subsubsection{\code{stop\_ymd}}

\begin{tabular}{lp{5.5in}}
Type    & integer \\
Default & none \\
Use     &  
Year, month and day of the simulation stop date.  The values are encoded
in an integer as \code{year*10000 + month*100 + day}.  If this value is
specified it takes precedence over both \code{nestep} and \code{nelapse}.
\end{tabular}

%................................................................................
\subsubsection{\code{stop\_tod}}

\begin{tabular}{lp{5.5in}}
Type    & integer \\
Default & 0 \\
Use     &  
Time of day (UTC) of the simulation stop date, in seconds.
\end{tabular}

%................................................................................
\subsubsection{\code{nestep}}

\begin{tabular}{lp{5.5in}}
Type    & integer \\
Default & none \\
Use     &  
If \code{nestep > 0} then \code{nestep} is the number of timesteps to
advance the solution past the start date.  If \code{nestep < 0} then the
stop date is \code{-nestep} days past the start date.  \code{nestep} is
ignored if \code{stop\_ymd} is set.
\end{tabular}

%................................................................................
\subsubsection{\code{nelapse}}

\begin{tabular}{lp{5.5in}}
Type    & integer \\
Default & none \\
Use     &  
If \code{nelapse > 0} then \code{nelapse} is the number of timesteps to
advance the solution past the start date (on an initial run) or past the
restart date (on a restart run).  If \code{nelapse < 0} then the stop date
is \code{-nelapse} days past the start or restart date (again depending on
run type).  \code{nelapse} is ignored if either \code{stop\_ymd} or
\code{nestep} is set.
\end{tabular}

%................................................................................
\subsubsection{\code{ref\_ymd}}

\begin{tabular}{lp{5.5in}}
Type    & integer \\
Default & from year, month, and day components of the start date \\
Use     &  
Year, month and day of the time coordinate's reference date.  The values
are encoded in an integer as \code{year*10000 + month*100 + day}.
\end{tabular}

%................................................................................
\subsubsection{\code{ref\_tod}}

\begin{tabular}{lp{5.5in}}
Type    & integer \\
Default & from the time of day component of the start date \\
Use     &  
Time of day (UTC) of the time coordinate's reference date, in seconds.
\end{tabular}

%................................................................................
\subsubsection{\code{perpetual\_ymd}}

\begin{tabular}{lp{5.5in}}
Type    & integer \\
Default & none \\
Use     &  
Perpetual date specified as \code{year*1000 + month*100 + day}.  This date
overrides the date from the initial file.  If running in ``aqua-planet''
mode then \code{perpetual\_ymd} is ignored and the perpetual date is set to
\code{321}.
\end{tabular}

%--------------------------------------------------------------------------------
\subsection{Application programmer interface}

This section provides an overview of the design and functionality of the
API.  The following is a summary of the time manager's public methods in
UML notation.  In UML a method prototype has the syntax ``\code{method-name
(argument-list) :return-type}'', where each argument in the comma separated
list is expressed as ``\code{intent arg-name:type}''.  The intent of an
argument is one of the keywords \code{in}, \code{out}, or \code{inout}.
Optional arguments are enclosed in brackets ``[ ]''.  For methods which
don't have a return value the ``\code{:return-type}'' is left off.

The API does not currently implement any alarm functionality because it is
not yet fully functional in the ESMF library.

%................................................................................
\setlength{\parindent}{0in}

%................................................................................
\subsubsection{\code{timemgr\_preset}}
\begin{verbatim}
subroutine timemgr_preset()
\end{verbatim}

\begin{quote}
\code{timemgr\_preset} is used for run-time initialization of namelist
variables.  Most namelist variables are statically initialized in the time
manager module.  This method is currently used only for \code{dtime} whose
default value depends on the dycore.  This method is provided because
\code{dtime} is used before the time manager is initialized (which happens
after the header of the initial conditions file is read to obtain the
default start date).
\end{quote}


%................................................................................
\subsubsection{\code{timemgr\_init}}
\begin{verbatim}
subroutine timemgr_init()
\end{verbatim}

\begin{quote}
\code{timemgr\_init} initializes the time manager module for an initial run.
Before \code{timemgr\_init} is called it is assumed that the namelist has
been read, and the date in the initial conditions file has been read.
\end{quote}


%................................................................................
\subsubsection{\code{timemgr\_restart}}
\begin{verbatim}
subroutine timemgr_restart()
\end{verbatim}

\begin{quote}
\code{timemgr\_restart} initializes the time manager module for a restart
run.  Before \code{timemgr\_restart} is called it is assumed that the
namelist has been read, and the restart data has been read from the restart
file by a call to \code{timemgr\_read\_restart}.
\end{quote}

%................................................................................
\subsubsection{\code{advance\_timestep}}
\begin{verbatim}
subroutine advance_timestep()
\end{verbatim}

\begin{quote}
\code{advance\_timestep} advances the time manager by one timestep.  This
includes updating the date and time information at the beginning and end of
the new timestep, and updating the alarms.
\end{quote}

%................................................................................
\subsubsection{\code{get\_step\_size}}
\begin{verbatim}
function get_step_size()
   integer :: get_step_size
\end{verbatim}

\begin{quote}
\code{get\_step\_size} returns the current timestep size in seconds.
\end{quote}

%................................................................................
\subsubsection{\code{get\_nstep}}
\begin{verbatim}
function get_nstep()
   integer :: get_nstep
\end{verbatim}

\begin{quote}
\code{get\_nstep} returns the current timestep number.
\end{quote}

%................................................................................
\subsubsection{\code{get\_curr\_date}}
\begin{verbatim}
subroutine get_curr_date(yr, mon, day, tod, offset)
   integer,           intent(out) :: yr, mon, day, tod
   integer, optional, intent(in)  :: offset
\end{verbatim}

\begin{quote}
\code{get\_curr\_date} returns the components of the date corresponding to
the end of the current timestep.  When the optional \code{offset} argument
is specified the current date is incremented by \code{offset} seconds (may
be negative).
The date components are: year (\code{yr}); month (\code{mon}); day of
month (\code{day}); and time of day in seconds past 0Z (\code{tod}).
\end{quote}

%................................................................................
\subsubsection{\code{get\_prev\_date}}
\begin{verbatim}
subroutine get_prev_date(yr, mon, day, tod)
   integer, intent(out) :: yr, mon, day, tod
\end{verbatim}

\begin{quote}
\code{get\_prev\_date} returns the components of the date corresponding to
the beginning of the current timestep.
The date components are: year (\code{yr}); month (\code{mon}); day of
month (\code{day}); and time of day in seconds past 0Z (\code{tod}).
\end{quote}

%................................................................................
\subsubsection{\code{get\_start\_date}}
\begin{verbatim}
subroutine get_start_date(yr, mon, day, tod)
   integer, intent(out) :: yr, mon, day, tod
\end{verbatim}

\begin{quote}
\code{get\_start\_date} returns the components of the date corresponding to
the starting date for the simulation.
The date components are: year (\code{yr}); month (\code{mon}); day of
month (\code{day}); and time of day in seconds past 0Z (\code{tod}).
\end{quote}

%................................................................................
\subsubsection{\code{get\_ref\_date}}
\begin{verbatim}
subroutine get_ref_date(yr, mon, day, tod)
   integer, intent(out) :: yr, mon, day, tod
\end{verbatim}

\begin{quote}
\code{get\_ref\_date} returns the components of the reference date part of
the time coordinate.
The date components are: year (\code{yr}); month (\code{mon}); day of
month (\code{day}); and time of day in seconds past 0Z (\code{tod}).
\end{quote}

%................................................................................
\subsubsection{\code{get\_perp\_date}}
\begin{verbatim}
subroutine get_perp_date(yr, mon, day, tod)
   integer, intent(out) :: yr, mon, day, tod
\end{verbatim}

\begin{quote}
\code{get\_perp\_date} returns the components of the perpetual date
corresponding to the end of the current timestep.
The date components are: year (\code{yr}); month (\code{mon}); day of
month (\code{day}); and time of day in seconds past 0Z (\code{tod}).
\end{quote}

%................................................................................
\subsubsection{\code{get\_curr\_time}}
\begin{verbatim}
subroutine get_curr_time(days, seconds)
   integer, intent(out) :: days, seconds
\end{verbatim}

\begin{quote}
\code{get\_curr\_time} returns the time at the end of the current
timestep.  \code{days} and \code{seconds} contain the components of the time
interval in units of days and seconds respectively.
\end{quote}

%................................................................................
\subsubsection{\code{get\_curr\_calday}}
\begin{verbatim}
function get_curr_calday(offset)
   real(r8)                      :: get_curr_calday
   integer, optional, intent(in) :: offset
\end{verbatim}

\begin{quote}
\code{get\_curr\_calday} returns the calendar day at the end of the current
timestep.  In perpetual mode the \code{get\_curr\_calday} returns the
perpetual calendar day.  When the optional \code{offset} argument is
specified the current date (or perpetual date) is incremented by
\code{offset} seconds (may be negative) before converting the date to a
calendar day.  The real kind, \code{r8}, specifies an 8-byte real value.
\end{quote}

%................................................................................
\subsubsection{\code{is\_first\_step}}
\begin{verbatim}
function is_first_step()
   logical :: is_first_step
\end{verbatim}

\begin{quote}
\code{is\_first\_step} returns true during the initialization phase of an
initial run.  This phase lasts from the point at which the time manager is
initialized until the first call to \code{advance\_timestep}.  In the CCM3
this phase was indicated by the conditional \code{if(nstep==0)...}
\end{quote}

%................................................................................
\subsubsection{\code{is\_first\_restart\_step}}
\begin{verbatim}
function is_first_restart_step()
   logical :: is_first_restart_step
\end{verbatim}

\begin{quote}
\code{is\_first\_restart\_step} returns true during the initialization
phase of a restart run.  This phase lasts from the point at which the time
manager is restart until the next call to \code{advance\_timestep}.  In the
CCM3 this phase was indicated by the conditional
\code{if(nstep==nrstrt)...}
\end{quote}

%................................................................................
\subsubsection{\code{is\_last\_step}}
\begin{verbatim}
function is_last_step()
   logical :: is_last_step
\end{verbatim}

\begin{quote}
\code{is\_last\_step} returns true during the final timestep of a run.
\end{quote}

%................................................................................
\subsubsection{\code{is\_end\_curr\_day}}
\begin{verbatim}
function is_end_curr_day()
   logical :: is_end_curr_day
\end{verbatim}

\begin{quote}
\code{is\_end\_curr\_day} returns true during the final timestep of each
day.  The final timestep of a day is the one whose ending date is equal to
or later than 0Z of the next day.
\end{quote}

%................................................................................
\subsubsection{\code{is\_end\_curr\_month}}
\begin{verbatim}
function is_end_curr_month()
   logical :: is_end_curr_month
\end{verbatim}

\begin{quote}
\code{is\_end\_curr\_month} returns true during the final timestep of each
month.  The final timestep of a month is the one whose ending date is equal to
or later than 0Z of the first day of the next month.
\end{quote}

%................................................................................
\subsubsection{\code{timemgr\_write\_restart}}
\begin{verbatim}
subroutine timemgr_write_restart(ftn_unit)
   integer, intent(in) :: ftn_unit
\end{verbatim}

\begin{quote}
\code{timemgr\_write\_restart} writes the state of the time manager to a
binary file attached to Fortran unit \code{ftn\_unit}.  It is assumed that
when running with MPI this method is only called from the master process.
\end{quote}

%................................................................................
\subsubsection{\code{timemgr\_read\_restart}}
\begin{verbatim}
subroutine timemgr_read_restart(ftn_unit)
   integer, intent(in) :: ftn_unit
\end{verbatim}

\begin{quote}
\code{timemgr\_read\_restart} reads the state of the time manager from a
binary file attached to Fortran unit \code{ftn\_unit}.  It is assumed that
when running with MPI this method is only called from the master process.
\end{quote}

%................................................................................
\setlength{\parindent}{\parindentorig}

%================================================================================
\section{Module design notes}

In discussions of valid ranges below we assume that the default native
integer is signed and at least 4 bytes long.

%--------------------------------------------------------------------------------
\subsection{Representation of dates}

A date is represented by 2 integer values.  One integer contains the
calendar date (year, month and day of month), and the other contains the
time of day (seconds past 0Z).

The year, month, and day of month components are packed in an integer using
the expression \code{year*1000 + month*100 + day}.  The range of valid
dates is -2147480101 through 2147481231 which corresponds to a valid range
of years from -214748 to 214748.

The time of day component of a date is represented as an integer number of
seconds.  Thus the precision of a date is 1 second.

The ESMF time manager defines a date type which may be initialized with
the 2 integer values used by CLM2.

%--------------------------------------------------------------------------------
\subsection{Representation of times}

Time values are represented by 2 integer values.  One integer contains the
number of days and the other contains the partial day in seconds.

The valid range of times (which are assumed to be positive) is 0 through
2,147,483,647 days + 86399 seconds, or about 5.9 million years.

The precision of a time is 1 second.

The ESMF time manager defines a time type which may be initialized with the
2 integer values used by CLM2.

%--------------------------------------------------------------------------------
\subsection{Counting timesteps}

The underlying philosophy of the ESMF time manager design is that
simulation control is based on the simulation date and the elapsed time
from some reference date.  There is no assumption that the timestep size is
constant.

In the CCM3 the timestep size is fixed and the simulation control is based
on counting timesteps.  The timestep number and size plus a reference date
are sufficient to compute the current date and time.

The ability to query the timestep number is for backwards compatibility
with earlier versions of CLM2 and allows a staged implementation of the time manager into
CLM2.  The alarm facilities in the time manager will eventually replace
the logic that depends on the timestep number (which implicitly assumes a
constant step size).

The timestep number in CLM2 is a native signed integer (usually 4 bytes)
which implies that only $2^{31}$ steps may be counted.  At the nominal
timestep size of 1200 seconds this imposes a valid range of 81,000 years on
the time manager.

%--------------------------------------------------------------------------------
\subsection{Setting the stop date}

The ESMF time manager determines if the current step is the last step of a
simulation by testing whether the current date is equal to or later than
the stop date.  To support the options provided by positive values of the
\code{nestep} and \code{nelapse} namelist variables, i.e., stopping the
simulation a specified number of timesteps past either the start or current
dates, the values must be converted to a corresponding stop date.  This is
done by incrementing the appropriate date by a time interval that is
calculated using the current value of the timestep size multiplied by the
specified number of timesteps.  This implementation assumes a constant
timestep size.

%--------------------------------------------------------------------------------
\subsection{Status}

\begin{itemize}

\item
The ESMF time manager has support for resetting the current date at an
arbitrary point in the simulation.  This feature has not yet been
incorporated into the CLM2 time manager.

\item
The ESMF time manager has not fully implemented the alarm functionality.

\end{itemize}

%================================================================================
\begin{thebibliography}{99}

\bibitem{esmf}
Earth System Modeling Framework (ESMF).\\
\begin{htmlonly}
\htmladdnormallink{http://www.esmf.ucar.edu/}
{http://www.esmf.ucar.edu/}.
\end{htmlonly}
\begin{latexonly}
http://www.esmf.ucar.edu/.
\end{latexonly}

\bibitem{coards}
``Conventions for the standardization of NetCDF files'',
Sponsored by the Cooperative Ocean/Atmosphere Research Data Service, a
NOAA/university cooperative for the sharing and distribution of global
atmospheric and oceanographic research data sets.  May 1995.\\
\begin{htmlonly}
\htmladdnormallink{http://ferret.wrc.noaa.gov/noaa_coop/coop_cdf_profile.html}
{http://ferret.wrc.noaa.gov/noaa_coop/coop_cdf_profile.html}.
\end{htmlonly}
\begin{latexonly}
http://ferret.wrc.noaa.gov/noaa\_coop/coop\_cdf\_profile.html.
\end{latexonly}

\bibitem{cf}
NetCDF Climate and Forecast (CF) Metadata Conventions.\\
\begin{htmlonly}
\htmladdnormallink{http://www.cgd.ucar.edu/cms/eaton/netcdf/CF-current.htm}
{http://www.cgd.ucar.edu/cms/eaton/netcdf/CF-current.htm}.
\end{htmlonly}
\begin{latexonly}
http://www.cgd.ucar.edu/cms/eaton/netcdf/CF-current.htm.
\end{latexonly}

\end{thebibliography}

\addcontentsline{toc}{section}{References}

\end{document}
