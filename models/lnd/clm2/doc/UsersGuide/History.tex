\section {History File Fields}
\label{sec_history_file}
The following sections discuss both the fields that may currently be
output to CLM2.1 history tapes as well as code modifications that the
user must make to add new user-defined fields to the history tapes.

\subsection {Model history fields}
Tables 9-16 list the fields that currently may be output to a CLM2.1
history tape. By default, these fields are also on the primary history
tape.  The dimensions of each field may include 'time' (days ince the
beginning of the simulation, 'levsoi' (number of soil layers, levsoi =
10) and 'lat' and 'lon' (number of latitude and longitude points,
e.g., lat=64, lon=128 for a T42 simulation), for grid averaged two
dimensional output and 'gridcell', 'landunit', 'column' or 'pft' for
one dimensional output.  Note that the 1d dimension type appearing in
the dimensions entry specifies only the default 1d output type. For
example, 'TSA' will be output by default in column 1d output. However,
that default type may be changed for a given history tape via the
setting of the namelist variable {\bf HIST\_TYPE1D\_PERTAPE}. The
Level column can contain either SL or ML as an entry, denoting a
single-level or multi-level field, respectively. Finally, unless
explicitly specified in the description, all fields are time averaged
over the requested history interval.

\begin{longtable}{|l|p{2.3in}|l|l|l|p{1.0in}|} 
\caption{\label{table_master_field_list} Master Field List - Temperature} \\
\hline
\endhead
\hline
Name & Description & Units & 1d Output & Level & Spatial Validity  \\ 
\hline	\hline	

{\bf TSA} 
& 2 m air temperature 
& K       
& column  
& SL 
& global  \\
\hline	

{\bf TV} 
& vegetation temperature 
& K       
& column  
& SL 
& global \\
\hline	

{\bf TG}
& ground temperature
& K        
& column
& SL 
& global \\
\hline	

{\bf TSOI}
& soil temperature 
& K       
& column  
& ML 
& lakes excluded \\  
\hline	

{\bf TLAKE} 
& lake temperature
& K        
& column
& ML 
& nonlakes excluded \\
\hline	

{\bf TSNOW}
& snow temperature 
& K         
& column
& SL 
& lakes excluded \\ 
\hline	

\end{longtable}

\begin{longtable}{|l|p{2.3in}|l|l|l|p{1.0in}|} 
\caption{\label{master_field_list_srad} Master Field List -Surface Radiation} \\
\hline
\endhead
\hline
Name & Description & Units & 1d Output & Level & Spatial Validity  \\ 
\hline	\hline	

{\bf FSA} 
& absorbed solar radiation   
& watt/m2        
& column
& SL
& global \\ 
\hline

{\bf FSR} 
& reflected solar radiation   
& watt/m2        
& column  
& SL
& global \\ 
\hline

{\bf NDVI} 
& surface normalized difference vegetation index   
& unitless        
& column  
& SL
& global \\ 
\hline

{\bf FIRA} 
& net infrared (longwave) radiation   
& watt/m2      
& column  
& SL
& global \\ 
\hline

{\bf FIRE} 
& emitted infrared (longwave) radiation   
& watt/m2      
& column
& SL
& global \\ \hline

\end{longtable}
      
\begin{longtable}{|l|p{2.3in}|l|l|l|p{1.0in}|} 
\caption{\label{master_field_list_sflux} Master Field List -Surface Energy Fluxes} \\
\hline
\endhead
\hline
Name & Description & Units & 1d Output & Level & Spatial Validity  \\ 
\hline	\hline	

{\bf FCTR} 
& canopy transpiration   
& watt/m2      
& column  
& SL
& global (set to 0 over lakes) \\
\hline

{\bf FCEV} 
& canopy (intercepted) evaporation   
& watt/m2      
& column  
& SL
& global (set to 0 over lakes) \\
\hline

{\bf FGEV} 
& ground evaporation   
& watt/m2      
& column  
& SL
& global  \\
\hline

{\bf FSH} 
& sensible heat   
& watt/m2      
& column  
& SL
& global \\
\hline

{\bf FGR} 
& heat flux into snow/soil (includes snow melt)   
& watt/m2      
& column  
& SL
& global \\
\hline

{\bf FSM} 
& snow melt heat flux   
& watt/m2      
& column  
& SL
& global \\
\hline

{\bf TAUX} 
& zonal surface stress   
& kg/m/s2      
& column  
& SL
& global \\
\hline

{\bf TAUY} 
& meridional surface stress   
& kg/m/s2      
& column  
& SL
& global \\ 
\hline

\end{longtable}
      
\begin{longtable}{|l|p{2.3in}|l|l|l|p{1.0in}|} 
\caption{\label{master_field_list_vegphen} Master Field List - Vegetation Phenology} \\
\hline
\endhead
\hline
Name & Description & Units & 1d Output & Level & Spatial Validity  \\ 
\hline	\hline	

{\bf ELAI} 
& exposed one-sided leaf area index   
& m2/m2      
& column  
& SL
& global \\
\hline 

{\bf ESAI} 
& exposed one-sided stem area index   
& m2/m2      
& column  
& SL
& global \\
\hline 

\end{longtable}
      
\begin{longtable}{|l|p{2.3in}|l|l|l|p{1.0in}|} 
\caption{\label{master_field_list_canopy} Master Field List - Canopy Physiology} \\
\hline
\endhead
\hline
Name & Description & Units & 1d Output & Level & Spatial Validity  \\ 
\hline	\hline	

{\bf RSSUN} 
& sunlit leaf stomatal resistance  (minimum over time interval)    
& s/m      
& column  
& SL
& excludes lakes \\
\hline 

{\bf RSSHA} 
& shaded leaf stomatal resistance (minimum over time interval)  
& s/m      
& column  
& SL
& excludes lakes \\
\hline 

{\bf BTRAN} 
& transpiration beta factor (soil moisture limitation)   
& unitless      
& column  
& SL
& excludes lakes \\
\hline 

{\bf FPSN} 
& photosynthesis   
& unitless      
& column  
& SL
& excludes lakes \\
\hline

\end{longtable}

\begin{longtable}{|l|p{2.3in}|l|l|l|p{1.0in}|} 
\caption{\label{master_field_list_hydro} Master Field List - Hydrology} \\
\hline
\endhead
\hline
Name & Description & Units & 1d Output & Level & Spatial Validity  \\ 
\hline	\hline	

{\bf H2OSOI} 
& volumetric soil water   
& mm3/mm3      
& column
& ML
& excludes lakes \\
\hline

{\bf H2OSNO} 
& snow depth (liquid water equivalent)   
& mm       
& column
& SL
& global \\
\hline

{\bf H2OCAN} 
& intercepted water   
& mm       
& column
& SL
& global (set to 0 over lakes) \\
\hline

{\bf SOILLIQ} 
& soil liquid water 
& kg/m2       
& column
& ML
& excludes lakes \\
\hline

{\bf SOILICE} 
& soil ice 
& kg/m2       
& column
& ML
& excludes lakes \\
\hline

{\bf SNOWLIQ} 
& snow liquid water 
& kg/m2         
& column
& SL
& excludes lakes \\
\hline

{\bf SNOWICE} 
& snow ice 
& kg/m2 
& column
& SL
& excludes lakes \\
\hline

{\bf SNOWDP} 
& snow height 
& m          
& column
& SL
& global \\
\hline

{\bf SNOWAGE} 
& snow age 
& unitless          
& column
& SL
& global (set to 0 over lakes) \\
\hline

{\bf QINFL} 
& infiltration 
& mm/s         
& column
& SL
& global \\
\hline

{\bf QOVER} 
& surface runoff 
& mm/s         
& column
& SL
& global \\
\hline

{\bf QRWGL} 
& surface runoff at glaciers, wetlands, lakes 
& mm/s         
& column
& SL
& global \\
\hline

{\bf QDRAI} 
& sub-surface drainage 
& mm/s         
& column
& SL
& global \\
\hline

{\bf QINTR} 
& interception 
& mm/s         
& column
& SL
& global (set to 0 over lakes) \\
\hline

{\bf QDRIP} 
& throughfall 
& mm/s         
& column
& SL
& global \\
\hline

{\bf QMELT} 
& snow melt 
& mm/s         
& column
& SL
& global \\
\hline

{\bf QSOIL} 
& ground evaporation 
& mm/s         
& column
& SL
& global \\
\hline

{\bf QVEGE} 
& canopy (intercepted) evaporation 
& mm/s         
& column
& SL
& global (set to 0 over lakes) \\
\hline

{\bf QVEGT} 
& canopy transpiration 
& mm/s         
& column
& SL
& global (set to 0 over lakes) \\
\hline

{\bf QCHOCNR} 
& RTM river discharge into ocean 
& m3/s         
& column
& SL
& global (only included if {\bf RTM} defined) \\
\hline

{\bf QCHANR} 
& RTM river flow (maximum subgrid flow) 
& m3/s         
& column
& SL
& global (only included if {\bf RTM} defined) \\
\hline

\end{longtable}

\begin{longtable}{|l|p{2.3in}|l|l|l|p{1.0in}|} 
\caption{\label{master_field_list_check} Master Field List - Water and Energy Balance Checks} \\
\hline
\endhead
\hline
Name & Description & Units & 1d Output & Level & Spatial Validity  \\ 
\hline	\hline	

{\bf ERRSOI} 
& soil/lake energy conservation error 
& watt/m2         
& column
& SL
& global \\
\hline

{\bf ERRSEB} 
& surface energy conservation error 
& watt/m2         
& column
& SL
& global \\
\hline

{\bf ERRSOL} 
& solar radiation conservation error 
& watt/m2         
& column
& SL
& global \\
\hline

{\bf ERRH2O} 
& total water conservation error 
& mm         
& column
& SL
& global \\
\hline

\end{longtable}

\begin{longtable}{|l|p{2.3in}|l|l|l|p{1.0in}|} 
\caption{\label{master_field_list_atm} Master Field List - Atmospheric Forcing} \\
\hline
\endhead
\hline
Name & Description & Units & 1d Output & Level & Spatial Validity  \\ 
\hline	\hline	

{\bf RAIN} 
& rain 
& mm/s         
& gridcell
& SL
& global \\
\hline

{\bf SNOW} 
& snow 
& mm/s         
& gridcell
& SL
& global \\
\hline

{\bf TBOT} 
& atmospheric air temperature 
& K         
& gridcell
& SL
& global \\
\hline

{\bf WIND} 
& atmospheric wind velocity magnitude 
& m/s         
& gridcell
& SL
& global \\
\hline

{\bf THBOT} 
& atmospheric air potential temperature 
& K         
& gridcell
& SL
& global \\
\hline

{\bf QBOT} 
& atmospheric specific humidity 
& kg/kg         
& gridcell
& SL
& global \\
\hline

{\bf ZBOT} 
& atmospheric reference height 
& m         
& gridcell
& SL
& global \\
\hline

{\bf FLDS} 
& incident longwave radiation 
& watt/m2         
& gridcell
& SL
& global \\
\hline

{\bf FSDS} 
& incident solar radiation 
& watt/m2         
& gridcell
& SL
& global \\
\hline

\end{longtable}

Note that for snow related fields (e.g. SNOWLIQ), horizontal averaging
is done only using columns that have snow. In this horizontal
averaging lake subgrid points are excluded. Furthermore, for 
snow related fields, vertical averaging is done by summing only over
valid snow layers.

\subsection{Adding new history fields}
\label{sec_historymod}

Model history output may appear in two-dimensional grid form or
one-dimensional subgrid form (depending on the value of the namelist
variable {\bf HIST\_DOV2XY}).  One-dimensional subgrid output may in
turn appear in gridcell, landunit, column or pft form.  History file
output is controlled by the files {\bf histFileMod.F90} and {\bf
clmpoint.F90} (in directory {\bf main/}).  The user must modify these
two files accordingly in order to add new user-defined history fields
to the history tapes. It is assumed in the following that the user is
completely familiar with Fortran 90 pointer concepts and syntax.

Module {\bf clmpoint.F90} creates arrays of one-dimensional real
pointers for selected {\bf clmtype} components.
Each of these arrays has one of the following forms:
\begin{verbatim}
real pointers to single level subgrid components:  
      clmptr_rs(derived_type_name)%val(one_dimensional_index)%p  
\end{verbatim}

\begin{verbatim}
real pointers to multi level subgrid components:  
      clmptr_ra(derived_type_name)%val(one_dimensional_index)%p
\end{verbatim}

The ``{\tt derived\_type\_name}'' specifies the particular {\bf
clmtype} derived type component that will be pointed to.  The ``{\tt
one\_dimensional\_index}'' spans the number of model gridcells,
landunits, columns or pfts that is appropriate for the {\tt
derived\_type\_name}.  At the top of {\bf clmpoint.F90} a default list
of ``{\tt derived\_type\_name}'' variable declarations are provided.
For example, {\tt ip\_pes\_t\_ref2m} corresponds to the ``{\tt
derived\_type\_name}'' for the array containing pointers to all pft
subgrid level {\tt t\_ref2m} values and {\tt one\_dimensional\_index}
can have any value from 1 to the number of model subgrid level
pfts. These variables are initialized in the call to routine
{\bf init\_pointer\_indices()} in subroutine {\bf clmpoint\_init}.
Note that the default list of ``{\tt derived\_type\_name}''
variables is much larger than those that are actually used in {\bf
clmpoint.F90} to set up the pointer arrays.

As a simple illustration, the following example code summarizes how
pointer arrays are created and used in {\bf clmpoint.F90} (refer to
files {\bf clmtype.F90} and {\bf clm\_mapping.F90} for details on {\bf
clmtype} components).
\begin{verbatim}
       ! Define clmpointer data structure
     
       type value_pointer_rs
          real(r8), pointer :: p
       end type value_pointer_rs
       type value_pointer_ra
          real(r8), dimension(:), pointer:: p
       end type value_pointer_ra
       type clmpoint_rs
          type (value_pointer_rs), dimension(:), pointer :: val
       end type clmpoint_rs
       type clmpoint_ra
          type (value_pointer_ra), dimension(:), pointer :: val
       end type clmpoint_ra

       integer, parameter :: max_mapflds = 500
       type (clmpoint_rs) :: clmptr_rs(max_mapflds)
       type (clmpoint_ra) :: clmptr_ra(max_mapflds)
            
       ! Determine ``derived_type_name'' for t_ref2m at the pft level
     
       integer :: ip_pes_t_ref2m 

       ! Initialize value for ip_pes_t_ref2m (along with all other
       ! ``derived_type_name'' variables

       call init_pointer_indices()
     
       ! Determine beginning and ending pft 1d indices for the
       ! current task (see clmtype.F90)
       ! pft 1d indices correspond to a one-dimensional vector 
       ! representation of the model subgrid level pft components
     
       begp = pfts1d%beg
       endp = pfts1d%end
     
       ! Allocate memory for t_ref2m pointer array
     
       allocate (clmptr_rs(ip_pes_t_ref2m)%val(begp:endp))
     
       ! Set up clmpointer array for t_ref2m
       ! ``g'' is a pointer to a model gridcell component
       ! ``l'' is a pointer to a landunit component within gridcell ``g''	
       ! ``c'' is a pointer to a column component within landunit ``l''	
       ! ``p'' is a pointer to a pft component within column ``c''	
       ! ``p%pps'' is the physical state component of pft ``p''
       ! ``p%pps%index1d'' is an index into a one dimensional vector
       ! representation of model subgrid level pft components 
     
       ! Loop over gridcells
       do gi = 1,clm%mps%ngridcells
          g => clm%g(gi)
     
          ! Loop over all landunits in the given gridcell
          do li = 1,g%gps%nlandunits
             l => g%l(li)
     
             ! Loop over all columns in the given landunit
             do ci = 1,l%lps%ncolumns
                c => l%c(ci)
     
                ! Loop over all pfts in the given column
                do pi = 1,c%cps%npfts
                   p => c%p(pi)
                   pindex = p%pps%index1d
     
                   ! Set up pft level pointer array to t_ref2m
                   clmptr_rs(ip_pes_t_ref2m)%val(pindex)%p => p%pes%t_ref2m
                end do ! end of PFTs loop

             end do ! end of columns loop
          end do ! end of landunits loop
       end do ! end of gridcells loop
\end{verbatim}

In the above example, {\tt pindex} spans the total number of model
subgrid level pfts for a given task (if MPI is used) or the total
number of model subgrid level pfts (if MPI is not used).  Setting up
the above pointer array is analogous to gathering all the 2m reference
temperatures from all the model subgrid level pfts into a single
one-dimensional array.  History file output is done {\bf ONLY} via
these one dimensional pointer arrays set up in {\bf clmpoint.F90}.

Module {\bf histFileMod.F90} contains routines that create and write
model history files and that update the model history buffer during
the course of the model simulation.  The initialization of history
fields is done in subroutine {\bf masterlist\_build()}, whereas the
determination of the history fields which are active by default on the
various history tapes is performed in subroutine {\bf
masterlist\_change\_active()}.

In {\bf masterlist\_build()}, two entries must be specified for each
field: an array {\tt hpindices} followed by a call to subroutine {\bf
masterlist\_addfld()}. The array, {\tt hpindices}, contains four
entries: hpindices(1) corresponds to the gridcell level {\tt
derived\_type\_name} for the requested field, {\tt hpindices(2)} is
the landunit subgrid level {\tt derived\_type\_name} for the field,
{\tt hpindices(3)} is the column subgrid level {\tt
derived\_type\_name} and {\tt hpindices(4)} is the pft subgrid level
{\tt derived\_type\_name}.  If the value of an {\tt hpindices} element
is set to {\tt -1}, one-dimensional output is not permitted for that
subgrid type for the given field. Requesting such output will result
in model termination.  If the value of an {\tt hpindices} element is
set to {\tt not\_valid}, then requesting one-dimensional output for
that subgrid type will result in a value of {\tt 1.e36} appearing for
all the output field values. 

As an example, {\tt hpindices} is set as follows for history field
{\tt SNOWAGE}:
\begin{verbatim}
       hpindices = (/-1, -1, ic_cps_snowage, not_valid/)
\end{verbatim}
Consequently, the user may not request one-dimensional gridcell or
landunit output for {\tt SNOWAGE}.  If the user requests
one-dimensional pft level output for {\tt SNOWAGE} via the following
namelist settings,
\begin{verbatim}
       HIST_FINCL2 = 'SNOWAGE' 
       HIST_TYPE1D_PERTAPE(2) = 'PFTS' 
\end{verbatim}
each output pft value for {\tt SNOWAGE} will be 1.e36. Since {\tt
SNOWAGE} is computed in the model as a column property, it makes sense
to set its pft {\tt derived\_type\_name} to {\tt not\_valid}.

As a second example, the history field, {\tt TSA} (2m reference
temperature) has {\tt hpindices} set as follows:
\begin{verbatim}
       hpindices = (/-1, -1, ic_ces_pes_a_t_ref2m, ip_pes_t_ref2m/)
\end{verbatim}
This implies that one dimensional history output for TSA can occur
either on columns or pfts.  Gridcell and landunit one dimensional
output for this field is not permitted due to the value of {\tt -1}.
Furthermore, one dimensional column output will use the array
\begin{verbatim}
       clmptr_rs(ic_ces_pes_a_t_ref2m)%val(:)%p.
\end{verbatim}
whereas one dimensional pft output will use the array
\begin{verbatim}
       clmptr_rs(ip_pes_t_ref2m)%val(:)%p.
\end{verbatim}

Following the setting of {\tt hpindices}, a call to subroutine {\bf
masterlist\_addfld()} must be made specifying required initialization
information for a given history field. For example, the following call
is made to initialize the 2m reference temperature:
\begin{verbatim}
       call masterlist_addfld (fname='TSA', type1d='column', units='K', numlev=1, &
              avgflag='A', long_name='2m air temperature', hpindices=hpindices)        
\end{verbatim}
The arguments to the above call are as follows:
\begin{verbatim}
       fname     : name of history field
       type1d    : default one dimensional output type 
                   (valid values are ``gridcell'',''landunit'',''column'' or ``pft'')
       units     : field units
       numlev    : number of vertical values
       avgflag   : default time averaging flag
                   (valid values are ``A''(average), ''I''(instantaneous),
                   ''M''(minimum) or ``X''(maximum))
       long_name : descriptive name of history field
       hpindices : array of one dimensional output parameter names.
\end{verbatim}
It is important to note that if the namelist variable {\bf
HIST\_DOV2XY} is set to true, history output will appear in two
dimensional grid form.  Two-dimensional grid output will be obtained
by calculating grid cell averages from the ``type1d'' subgrid
components.
 
The following lists steps that must be taken by the user to add a new
history field to the model. Currently, the upward longwave radiation
above the canopy is not output to the history file. We assume that the
user wants to add this field to the history output at either the
column or pft subgrid level. The following steps assume that a
clmpointer array does not exist for the desired history
variable. Steps 1 and 2 below discuss how such a pointer array is
created. If a clmpointer array already exists for the needed history
variable then only steps 4 and 5 are needed. Step 3 is needed only if
1d output is requested at a subgrid level where spatial averaging is
required and that spatial averaging is not currently in the code.

\begin{enumerate}

\item
Currently, parameter values for the {\tt derived\_type\_names} of
column and pft level upward longwave radiation are already contained
at the top of {\bf clmpoint.F90} ({\tt ip\_pef\_ulrad} and {\tt
ic\_cef\_pef\_a\_ulrad}). If these parameters did not already exist,
they would have to be added to the list of existing parameter values.

\item
The following code fragments should be added to subroutine {\bf
clmpoint\_init()} in module {\bf clmpoint.F90}.
\begin{verbatim}
! Add code to allocate memory for pft level pointer array 
! for upward longwave radiation

allocate (clmptr_rs(ip_pef_ulrad)%val(begp:endp))

! Add code to allocate memory for column level pointer array 
! for upward longwave radiation

allocate (clmptr_rs(ic_cef_pef_a_ulrad)%val(begc:endc)) 

! Add code to set up column and pft level clmpointer arrays for
! direct surface albedo history output
! ``g'' is a pointer to a model gridcell component
! ``l'' is a pointer to a landunit component within gridcell ``g''	
! ``c'' is a pointer to a column component within landunit ``l''	
! ``c%cps'' is the physical state component of column ``c''
! ``c%cps%index1d'' is an index into a one dimensional vector
! representation of model subgrid level column components 
! ``p'' is a pointer to a pft component within column ``c''	
! ``p%pps'' is the physical state component of pft ``p''
! ``p%pes'' is the energy state component of pft ``p''
! ``p%pps%index1d'' is an index into a one dimensional vector
! representation of model subgrid level pft components 

! Loop over gridcells
do gi = 1,clm%mps%ngridcells
   g => clm%g(gi)

   ! Loop over all landunits in the given gridcell
   do li = 1,g%gps%nlandunits
      l => g%l(li)

     ! Loop over all columns in the given landunit
      do ci = 1,l%lps%ncolumns
         c => l%c(ci)
         cindex = c%cps%index1d

         ! Set up column level pointer array to albd
         clmtr_rs(ic_cef_pef_a_ulrad)%val(cindex)%p => c%cef%pef_a%ulrad

         ! Loop over all pfts in the given column
         do pi = 1,c%cps%npfts
            p => c%p(pi)
            pindex = p%pps%index1d

            ! Set up pft level pointer array to albd
            clmptr_rs(ip_pef_ulrad)%val(pindex)%p => p%pef%ulrad

         end do ! end of PFTs loop
      end do ! end of columns loop
   end do ! end of landunits loop
end do ! end of gridcells loop
\end{verbatim}

\item
Currently, the model calculates the upward longwave radiation at
both the pft and column subgrid level. Pft level values are obtained
in module {\bf CanopyFluxesMod.F90} whereas column level values are
calculated in module {\bf pft2columnMod.F90}.

\item
The following statements should be added to subroutine {\bf
masterlist\_build()} (in module {\bf histFileMod.F90}):
\begin{verbatim}
hpindices = (/-1, -1, ic_cef_pef_a_ulrad, ip_pef_ulrad/)
call masterlist_addfld (fname='ULRAD', type1d='pft', units='W/m2',
       numlev=1, avgflag='A', long_name='upward longwave radiation above canopy', 
       hpindices=hpindices)
\end{verbatim}
Note that setting type1d to 'pft' above will result in default 1d
output on the pft subgrid level for the history field 'ULRAD'.
\item
The following statement should be added to subroutine {\bf
masterlist\_change\_active()} (in module {\bf histFileMod.F90}) if the
user wants this field on the primary tape by default:
\begin{verbatim}
call masterlist_make_active (name='ULRAD', tape_index=1)
\end{verbatim}

\end{enumerate}
