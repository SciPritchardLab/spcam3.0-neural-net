\section {Testing Model Changes}

The script test-model.pl in the {\bf bld/offline/tests} directory runs
a suite of basic tests for the CLM2.1 model running in offline mode on
a T42 model grid.  In order to use test-model.pl, users need to
have at least Perl version 5.004 on their system.  This test suite is
designed for model testing during the modification of model code or
for simply porting the code to another machine. The test suite
provides a check that the basic functionality of the model still works
despite the changes that have been introduced.

In this section, usage of test-model.pl for basic acceptance testing
is described.  The command-line arguments to ``test-model.pl'' and
environment variables useful to the script are described. The script
is designed to be run interactively.  Although this script can be run
at other labs, testing has only been done on NCAR
machines. Consequently, the following discussion will be confined to
NCAR platforms. The user should feel free to try the scripts at other
labs.

Before running test-model.pl, the user needs to set the environment
variable, {\bf CSMDATA}, to the full disk pathname containing the
untarred data subdirectories.  At NCAR, CCSM input datasets are stored
on a NFS mounted directory,
``/fs/cgd/csm/inputdata/lnd/clm2''. Consequently, this is the default
value for \$CSMDATA in the testing scripts.  If the user has untarred
the source code such that the test-model.pl script is in the directory
``/home/user/clm2/bld/offline/tests'', then the user may invoke the
script as follows:\newline

\begin{verbatim}
     cd /home/user/clm2/bld/offline/tests 
     test-model.pl 
\end{verbatim}

The script, test-model.pl, is designed such that the common settings
the user might want to control are easily set by command line
arguments. The ``-h'' option to test-model.pl lists all of the
possible command-line options. 

\begin{verbatim}
    -h       = Help (this message) 
    -t       = List the tests that are performed  
    -clean   = Clean the old directories out   
    -nofail  = Continue even if errors are found   
    -res     = Resolution (T42)
    -s n     = Skip to given test n (or range of numbers)   
             = (example -s 4 start with test no. 4)   
             = (or      -s 2-4 do tests 2 through 4)   
    -c dir   = Compare to another version of the CLM2 model in this directory   
             = (example -c /home/user/clm2/src)  
    -l lab   = Set the lab you are running at    
\end{verbatim}

The ``-t'' option to ``test-model.pl'' lists all of the possible
tests.  Below is a brief summary of each test. \newline

\medskip
\begin{itemize}
\item 
Tests 1 and 2. Run three time-steps with DEBUG compiler flags on, with
SPMD on and off, and then ensure that answers are identical.
\item 
Tests 3,4 and 5. Perform an initial run followed by a restart using
fewer SPMD tasks (if SPMD enabled), and fewer threads. Then do an
initial run the same number of timesteps as the restart, and compare
answers ensuring they are identical.

\item 
Test 6. If a comparison to a previous code library is requested (via
the ``-c'' option), repeat the last initial run with the previous code
library and check if answers are identical.
\end{itemize}
\medskip

An important feature of test-model.pl is the ability to compare
user modified model code to a previous program library. This is useful in
order to ensure that the implemented modifications do not change
answers if that is what is expected.  Using the command line option
``-c dir'' the user can compare to a previous program library by giving
the full path to the root of the library to compare to. For example,
if a test library can be found in ``/home/user/clm2mod/src'' and the
unmodified library is located in ``/home/user/clm2/src'' then ``-c''
can be used as follows: \newline

\begin{verbatim}
     cd /home/user/testmod/bld/offline/tests 
     test-model.pl -c /home/user/clm2/src 
\end{verbatim}

Many times differences with respect to a control library are intended
to be bit-for-bit.  When the ``-c'' option is used, only test 6 is
compared to test 5 (see the list of tests above).  If this comparison
is identical, the modified model is identified as being bit-for-bit
with the control library. Whereas test-model.pl can easily identify if
two model libraries give identical answers, it is more difficult to
verify if changes are within machine roundoff. Currently, if non
bit-for-bit differences occur, differences between the history files
produced by both the control as well as modified libraries must be
examined to determine if these differences indicate only roundoff
level changes.
