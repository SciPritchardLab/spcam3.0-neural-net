\documentclass[]{article}
\usepackage{html}
\usepackage[T1]{fontenc}
\usepackage{longtable}

\textwidth 6.5in
\textheight 8.5in
\addtolength{\oddsidemargin}{-.75in}

\begin{document}
\bodytext{BGCOLOR=white LINK=#083194 VLINK=#21004A}

\begin{titlepage}

\begin{latexonly}
\noindent {\bf Community Climate System Model} \\
\noindent National Center for Atmospheric Research, Boulder, CO \\
\vspace{2in}
\end{latexonly}

\begin{center}
{\Large\bf CLM2.0 Code Reference} \\
{\bf Version 1.0}
\medskip
{\it Mariana Vertenstein, Sam Levis, Keith Oleson, and Peter Thornton}
\end{center}

\end{titlepage}

\tableofcontents

\newpage
\section{Code Structure}

\subsection {Calling Tree}

The following is a brief outline of the calling sequence for the main
CLM2.0 driver routine, driver.F90.  A comprehensive outline of the
calling sequence for the full CLM2.0 model is provided in the
accompanying html document,
\htmladdnormallink{CLM2.0 Comprehensive Calling Sequence}
{clm_offline_tree.html}.

\begin{itemize}
\item ->  histend 
\item ->  get\_curr\_date 
\item ->  interpMonthlyVeg 
  \begin{itemize}
  \item ->  readMonthlyVegetation 
  \end{itemize}
\item -> {\bf ***begin first loop over patch points**} 
\item ->  Hydrology1 
  \begin{itemize}
  \item ->  Fwet 
  \end{itemize}
\item ->  Biogeophysics1 
  \begin{itemize}
  \item ->  QSat 
  \item ->  SurfaceRadiation 
  \item ->  BareGroundFluxes 
    \begin{itemize}
    \item ->  MoninObukIni 
    \item ->  FrictionVelocity 
    \end{itemize}
  \item ->  CanopyFluxes 
     \begin{itemize}
     \item ->  Qsat 
     \item ->  MoninObukIni 
     \item ->  FrictionVelocity 
     \item ->  Stomata (call for sunlit leaves and shaded leaves) 
     \item ->  SensibleHCond 
     \item ->  LatentHCond 
     \item ->  Qsat 
     \end{itemize}
  \end{itemize}
\item ->  Biogeophysics\_Lake 
  \begin{itemize}
  \item ->  SurfaceRadiation 
  \item ->  Qsat 
  \item ->  MoninObukIni 
  \item ->  FrictionVelocity 
  \item ->  Qsat 
  \item ->  Tridiagonal 
  \end{itemize}
\item ->  Biogeochemistry? 
\item ->  EcosystemDyn 
\item ->  SurfaceAlbedo 
  \begin{itemize}
  \item ->  shr\_orb\_decl 
  \item ->  shr\_orb\_cosz 
  \item ->  SnowAlbedo (called for direct beam and diffuse beam) 
  \item ->  SoilAlbedo 
  \item ->  TwoStream (called for visible direct, visible diffuse, NIR direct and NIR diffuse) 
  \end{itemize}
\item ->  Biogeophysics2 
  \begin{itemize}
  \item ->  SoilTemperature 
    \begin{itemize}
    \item ->  SoilThermalProp 
    \item ->  Tridiagonal 
    \item ->  PhaseChange 
    \end{itemize}
  \end{itemize}
\item -> {\bf ***end first loop over patch points***} 
\item -> {\bf ***begin second loop over patch points***} 
\item ->  Hydrology2 
   \begin{itemize}
   \item ->  SnowWater 
   \item ->  SurfaceRunoff 
   \item ->  Infiltration 
   \item ->  SoilWater 
     \begin{itemize}
     \item -> Tridiagonal 
     \end{itemize}
   \item ->  Drainage 
   \item ->  SnowCompaction 
   \item ->  CombineSnowLayers 
     \begin{itemize}
     \item -> Combo 
     \end{itemize}
   \item ->  DivideSnowLayers 
     \begin{itemize}
     \item -> Combo 
     \end{itemize}
   \item -> WetIceHydrology 
   \end{itemize}
\item ->  Hydrology\_Lake 
\item ->  SnowAge 
\item ->  BalanceCheck 
\item -> {\bf ***end second loop over patch points***} 
\item ->  histUpdate 
\item ->  Rtmriverflux 
   \begin{itemize}
   \item ->  Rtm 
   \end{itemize}
\item ->  histhandler 
\item ->  restwrt 
\item ->  inicwrt 

\end{itemize}

\subsection {Code Flow - Main Interface}

The CLM2.0 model can be built to run in one of three modes. It can run
as a stand alone executable where atmospheric forcing data is
periodically read . This will be referred to as offline mode. It can
also be run as part of the Community Atmosphere Model (CAM) where
communication between the atmospheric and land models occurs via
subroutine calls. This will be referred to as cam mode. Finally, it
can be run as a component in a system of geophysical models (CCSM).
In this mode, the atmosphere, land (CLM2), ocean and sea-ice models
are run as separate executables that communicate with each other via
the CCSM flux coupler. This will be referred to as csm mode. \newline

\noindent {\bf offline mode:} The routine, {\bf program\_off.F90},
provides the program interface for running CLM2.0 in offline
mode. This routine first initializes the CLM2.0 model. Part of this
initialization consists of the the determination of orbital parameters
by a call to the routine {\bf shr\_orb\_params.F90}.  Subsequently,
the time stepping loop of the model is executed by obtaining
atmospheric forcing and calling the CLM2.0 driver. \newline

\noindent {\bf csm mode:} The routine, {\bf program\_csm.F90},
provides the program interface for running CLM2.0 in csm mode. In this
mode, orbital parameters are obtained from the flux coupler during
initialization whereas atmospheric data are obtained from the flux
coupler during the time stepping loop.
\newline

\noindent {\bf cam mode:} The module, {\bf atm\_lndMod.F90}, contains
the subroutine interfaces necessary to run CLM2.0 in cam mode. The
model is initialized by a call to subroutine {\bf atmlnd\_ini} whereas
subroutine {\bf atmlnd\_drv} is called at every time step by the
CAM atmospheric model to update the land state and return the 
necessary states and fluxes back to the atmosphere.
\newline

\subsection {Code Flow - Driver Loop}

\noindent 
The following presents a brief outline of the routines appearing in
the calling sequence for {\bf program\_off.F90}, the main CLM2.0 offline
program. Most of the following routines are invoked from the driver routine,
{\bf driver.F90}.

\medskip

\noindent {\bf shr\_orb\_params:} 
Calculate Earth's orbital parameters. \newline

\noindent {\bf Initialize:} 
Calls a series of subroutines (see \htmladdnormallink{CLM2.0
Comprehensive Calling Sequence} {clm_offline_tree.html}) which
initialize model parameters, read and/or create a surface dataset,
read the initial file (initial simulations only), read the restart
file (restart or branch simulations only). If no initial dataset is
specified in the namelist, the model uses an internal
initialization. If no surface dataset is specified in the namelist,
the model uses a list of raw datasets to create a surface dataset and
read it in. \newline

\noindent {\bf atmdrv:} 
Read in atmospheric fields and generate atmospheric forcing. \newline

\noindent {\bf driver:} 
Driver for CLM2.0 physics. \newline

\noindent {\bf histend:} 
Determine if current time step is the end of history interval. \newline

\noindent {\bf get\_curr\_date:} 
Determine calender information for next time step. \newline

\noindent {\bf interpMonthlyVeg:} 
Determine if two new months of vegetation data need to be read in. \newline

\noindent {\bf readMonthlyVegetation:} 
Read monthly vegetation data for two consecutive months. \newline

\noindent {\bf Hydrology1:} 
Calculation of (1) water storage of intercepted precipitation (2)
direct throughfall and canopy drainage of precipitation (3) the
fraction of foliage covered by water and the fraction of foliage that
is dry and transpiring and (4) snow layer initialization if the snow
accumulation exceeds 10 mm. \newline

\noindent {\bf Fwet:} 
Determine the fraction of foliage covered by water and the fraction of
foliage that is dry and transpiring. \newline

\noindent {\bf Biogeophysics1:} 
Main subroutine to determine leaf temperature and surface fluxes based
on ground temperature from previous time step. \newline

\noindent {\bf QSat:} 
Compute saturation mixing ratio and the change in saturation mixing
ratio with respect to temperature. \newline

\noindent {\bf SurfaceRadiation:} 
Compute Visible and NIR solar fluxes absorbed by vegetation and ground
surface. Split canopy absorption into sunlit and shaded
canopy. Calculate NDVI and reflected solar radiation. This routine is
also used for surface radiation for lake biogeophysics. \newline

\noindent {\bf BareGroundFluxes:}
Compute sensible and latent heat fluxes and their derivatives with
respect to ground temperature using ground temperatures from
previous time step for non-vegetated surfaces or snow-covered
vegetation. Calculate stability and aerodynamic resistances. \newline

\noindent {\bf MoninObukIni:}
Initialize Monin-Obukhov length. \newline

\noindent {\bf FrictionVelocity:}
Calculation of the friction velocity and the relation for potential
temperature and humidity profiles of surface boundary layer. \newline

\noindent {\bf CanopyFluxes:}
Calculates the leaf temperature, leaf fluxes, transpiration,
photosynthesis and updates the dew accumulation due to
evaporation. \newline

\noindent {\bf Stomata:}
Leaf stomatal resistance and leaf photosynthesis. Uses Ball-Berry
formulation for stomatal conductance, and Farquhar photosynthesis
model. \newline

\noindent {\bf SensibleHCond:}
Provides dimensional and non-dimensional sensible heat conductances
for canopy and soil flux calculations. \newline

\noindent {\bf LatentHCond:}
Provides dimensional and non-dimensional latent heat conductances for
canopy and soil flux calculations. \newline

\noindent {\bf Biogeophysics\_Lake:}
Calculates lake temperatures and surface fluxes. Lake temperatures
are determined from a one-dimensional thermal stratification model
based on eddy diffusion concepts to represent vertical mixing of
heat. \newline

\noindent {\bf EcosystemDyn:}
Determine vegetation phenology \newline

\noindent {\bf SurfaceAlbedo:}
Surface albedos, fluxes (per unit incoming direct and diffuse
radiation) reflected, transmitted, and absorbed by vegetation, and
sunlit fraction of the canopy. \newline

\noindent {\bf shr\_orb\_decl:}
Determine solar declination for next time step. \newline

\noindent {\bf shr\_orb\_cosz:}
Determine cosine of solar zenith angle for next time step. \newline

\noindent {\bf SnowAlbedo:}
Determine direct and diffuse visible and NIR snow albedos. \newline

\noindent {\bf SoilAlbedo:}
Determine soil/lake/glacier/wetland albedos. \newline

\noindent {\bf TwoStream:}
Use two-stream approximation to calculate visible and NIR fluxes
absorbed by vegetation, reflected by vegetation, and transmitted
through vegetation for unit incoming direct or diffuse flux given an
underlying surface with known albedo. \newline

\noindent {\bf Biogeophysics2:}
Main subroutine to determine soil/snow temperatures including ground
surface temperature and update surface fluxes for new ground
temperature. \newline

\noindent {\bf SoilTemperature:}
Determine soil/snow temperatures including ground surface
temperature. \newline

\noindent {\bf SoilThermProp:}
Determine soil/snow thermal conductivity and heat capacity. \newline

\noindent {\bf Tridiagonal:}
Solve tridiagonal system of equations. \newline

\noindent {\bf PhaseChange:}
Determine phase change within soil/snow layers. \newline

\noindent {\bf Hydrology2:}
Main subroutine to determine soil/snow hydrology. \newline

\noindent {\bf SnowWater:}
Determine the change of snow mass and the snow water. \newline

\noindent {\bf SurfaceRunoff:}
Determine surface runoff. \newline

\noindent {\bf Infiltration:}
Determine infiltration into surface soil layer (minus the
evaporation). \newline

\noindent {\bf SoilWater:}
Determine soil moisture. \newline

\noindent {\bf Drainage:}
Determine subsurface runoff. \newline

\noindent {\bf SnowCompaction:}
Determine natural compaction and metamorphosis of snow. \newline

\noindent {\bf CombineSnowLayers:}
Combine thin snow elements. \newline

\noindent {\bf Combo:}
Combine two snow elements in terms of temperature, liquid water and
ice contents, and layer thickess. \newline

\noindent {\bf DivideSnowLayers:}
Divide thick snow elements. \newline

\noindent {\bf WetIceHydrology:}
Calculate hydrology for ice and wetland.  Maintains a constant water
volume for wetlands and ice. \newline

\noindent {\bf Hydrology\_Lake:}
Determine fate of snow on lake. Force constant lake volume. \newline

\noindent {\bf SnowAge:}
Determine age of snow for albedo calculations. \newline

\noindent {\bf BalanceCheck:}
Error checks for energy and water balance. \newline

\noindent {\bf histUpdate:}
Accumulate history fields over history time interval. \newline

\noindent {\bf Rtmriverflux:}
Route surface and subsurface runoff into rivers for RTM river routing
model. \newline

\noindent {\bf Rtm:}
RTM river routing model. \newline

\noindent {\bf histHandler:}
Main history file handler. This code 1) increments field accumulation
counters at every time step and determines if next time step is
beginning of history interval 2) at the end of a history interval,
increments the current time sample counter, opens a new history file
if needed, writes history data to current history file and resets field
accumulation counters to zero and 3) when a history file is full, or
at the last time step of the simulation, closes history file and
disposes to mass store (only if file is open), resets time sample
counter to zero (only if file is full) and increments file counter by
one (only if file is full) \newline

\noindent {\bf restwrt:}
Write binary restart files. \newline

\noindent {\bf inicwrt:}
Write netCDF initial files. \newline

\section{Modifying CLM2.0 Input and Output Files}

\subsection {Modifying Surface Data}

A list of high resolution ``raw'' datasets is provided with the CLM2.0
distribution.  These raw datasets are only needed if a raw surface
dataset is to be created at runtime (see CLM2.0 User's Guide). The
following is a list of these datasets:

\begin{itemize}
\item mksrf\_pft.nc: raw vegetation type dataset 
\item mksrf\_soitex.10level.nc: raw soil texture dataset 
\item mksrf\_soicol\_clm2.nc: raw soil color dataset 
\item mksrf\_lanwat.nc: raw inland water dataset 
\item mksrf\_urban.nc: raw urban dataset 
\item mksrf\_glacier.nc: raw glacier dataset
\item mksrf\_lai.nc: raw leaf and stem area index, canopy top and bottom height dataset
\end{itemize}

The user has the option of replacing any of these raw datasets, as
long as the structure of the netCDF files is preserved. A series of
fortran routines to convert ascii to raw netCDF data is provided in
the directory, /tools/convert\_ascii.  Alternatively, the user may
wish to modify the model surface data directly as long as the original
netCDF file structure is preserved.

\subsection {History fields}

To output a history field which has not been included in the CLM2.0 code,
the user must do the following: \newline

\noindent 
Step 1. Add the appropriate initialization call for the required
history field in subroutine {\bf histlst}.  For example, 
the following lines could be added in {\bf histlst}:

\medskip
call histfldini(nflds, 'NEWFLD ', 'UNITS', nsing, naver,'FIELD DESC', .true., histfld) 
\newline

\noindent 
\begin{itemize}
\item where NEWFLD is the new field name. 
\item UNITS is the new field units. 
\item the level structure of the new history field is either ``nsing'' 
(for a single level field) or ``nsoil'' (for a multi-level ``soil''
field).
\item the type of time averaging desired is set to either
``naver'' (average), ``ninst'' (instantaneous), ``nmini'' (minimum)
or ``nmaxi`` (maximum)
\item FIELD DESC is the new field description
\item the field is specified to be active or non-active by default [.true., .false.] 
If value is .true. the field is active by default (ie, written to
history). \newline
\end{itemize}

\noindent 
Step 2. In subroutine {\bf histUpdate}, add the following lines (using
the following only as an example): \newline

\noindent !\$OMP PARALLEL DO PRIVATE (K)
\newline do k = begpatch,endpatch
\newline tmpslev(k) = call histslf('NEWFLD', tmpslev)
\newline end do

\noindent 
where tmpslev and histslf are used for adding a new single-level field
(whereas tmpmlev and histmlf are used for adding a new multi-level
field).  If adding a multi level field, the call to histmlf passes an
additional variable denoting the number of levels (nlevsoi or
nlevlak). It may also be necesasry to increase the maximum allowable
number of single level fields.  This can be done by increasing the
values of the parameters {\bf max\_slevflds} or {\bf max\_mlevflds} in
routine src/main/clm\_varpar.F90.

\subsection {Initial datasets}

Initial datasets are created by the land model periodically (for
details, see the CLM2.0 User's Guide). Initial datasets are netCDF
files containing instantaneous variable data. Since certain model
variables take relatively long to spin up (e.g. soil and snow related
variables), using an initial dataset ensures faster spin up of the
model than using arbitrary initialization.  The variables stored in
initial dataset files are dimensioned by number of landpoints (e.g.,
numland = 1956 for a T31 global resolution) and the maximum number of
subgrid patches in a land gridcell (maxpatch = 8).  If a variable has
multiple vertical levels, an additional dimension (e.g. nlevsoi = 10)
is used.

Changing an initial dataset should not be necessary most of the
time. However, if new variables must be added to the initial dataset,
the user should refer to subroutines {\bf inicwrt} and {\bf inicrd} in
the file {\bf src/main/inicFileMod.F90}. The user should start in
subroutine inicwrt by defining the new variable as a single or
multi-level field (exactly as is done for existing variables). In the
same subroutine the user should add a few lines to write out the new
variable (again as done for existing variables). Subsequently, the
user should add a similar group of lines to subroutine {\bf inicrd},
for the variable to be recognized by the code at
initialization. Comments in the code are clear for the user to
identify where variables are defined, where they are written, and
where they are read.

An important distinction between initial and restart files is that
initial files have a self-describing format (since they are netCDF),
and therefore may be used by future versions of the model to
initialize a simulation. Restart files (see below), on the other hand,
may change frequently during model development and may be difficult to
use with different versions of the code.

\subsection {Restart files}

Restart files contain instantaneous binary data for a set of
time-dependent variables. The data is stored in binary format to
ensure bit-for-bit agreement with a terminated run which was
subsequently restarted. Restart files contain a version number to
ensure that the code being used is compatible with the restart file
being read. If the user modifies the contents of the restart file, the
version number should be incremented.

Restart files should be used to extend previously started simulations
or to branch from one simulation to another in which history file
namelist variables have been modified. For a branch run, the user must
change the simulation's case name.

The criterion for a ``successful restart'' is that the model has
available exactly the same information upon restart that it would have
had if it had not stopped. Restart files may need to be modified
during model development to include new time-dependent variables.  A
simple rule is that if a run produces even slightly different answers
when restarted compared to when left uninterrupted, then certain
variables are missing from the restart file.

Restart files are written in subroutine {\bf restwrt} and read in
subroutine {\bf restrd} (which are contained in file
src/main/restFileMod.F90.  Subroutine {\bf restrd} reads the data in
exactly the same order as it was written out and the simulation
continues as though uninterrupted. If a user modifies the code such
that it becomes necessary to add a new variable to the restart file,
separate entries must be made in subroutines {\bf restwrt} and {\bf
restrd}. As an example, if the following field were to be added the
the restart file, the following two entries would have to be made:
\medskip

\noindent 
\newline restwrt: 
\newline buf1d(begpatch:endpatch) = clm(begpatch:endpatch)\%newvar
\newline call wrtout (nio, buf1d) 

\noindent 
\newline restrd: 
\newline call readin (nio, buf1d)
\newline clm(begpatch:endpatch)\%newvar = buf1d(begpatch:endpatch)
\newline

\noindent 
The user must make sure to place the former two lines in exactly the
same order with respect to existing variables as the latter two
lines. For example, variable fwet is between variables t\_grnd and
tlai in both subroutines {\bf restrd} and {\bf restwrt}.

\section{Error codes}

Error codes exist in the model to warn the user of inconsistencies and
errors. The model is released to the community as a package which
should not trigger such inconsistencies or errors. However, the user
may change the code or modify the input files in a way which triggers
such a message. The user may even simply operate the model on a
platform which brings up an error message. In many cases these error
codes are followed by a stoprun in order to alert the user of a
problem which will render the simulation useless or will cause it to
fail later.

In general, when an error message appears, the user must switch gears
from production to debugging mode. The error message may make the
corrective measures obvious. If not, the user will need to find the
line in the code which wrote the error message and gradually back
track till the cause for the error is found. The old fashioned
placement of write statements in the code or, if available, debugging
software may help understand the problem. When using debugging
software, it is useful to change DEBUG to .true. in the jobscript.

\subsection {Energy and water balance errors}

To make sure that the model conserves energy and mass, energy and
water imbalances appearing in a simulation will trigger errors and
halt the simulation,

To accomplish energy conservation, all energy inputs at the land
surface are reflected or absorbed. Absorbed energy may return to the
atmosphere as sensible heat or may be emitted as terrestrial
radiation, stored as soil heat, used to evaporate or transpire water,
or used to melt snow and ice. The error check for the energy balance
is in subroutine src/main/BalanceCheck.F90.

Similarly, water incident on the land will evaporate, transpire
through leaf stomata, run off and drain from the soil, or be stored in
the soil and snow pack. The error check for the water balance is in
subroutine BalanceCheck.

Water which runs off or drains from the soil fills river chanels and
flows downstream. At every model time step, the global sum of water
which runs off and drains from the soil into the rivers must equal the
global sum of change in river volume. This is because the global sum
of water flowing out of grid cells must cancel the global sum of water
flowing into grid cells from upstream. This error check is in
subroutine src/riverroute/RtmMod.F90.

\end{document}






