\section {Offline Mode Namelist Examples}
\label{sec_examples}

The following examples illustrate different namelist options that 
can be used to run CLM2.1 in offline mode.

\subsection {Example 1: Offline initial run, one day, global}

When the model is run in offline mode using a pre-existing surface
dataset, the minimum namelist parameters are: {\bf CASEID}, {\bf
NSREST}, {\bf NESTEP} or {\bf NELAPSE}, {\bf FSURDAT}, {\bf FPFTCON},
{\bf OFFLINE\_ATMDIR}, {\bf START\_YMD}, and {\bf DTIME}. If {\bf
FSURDAT} is blank, a surface dataset will be generated at run time and
additional variables need to be specified (see section
\ref{subsec_model_input_data_namelist} and Examples 5 and 6).  All other
namelist parameters will be set to default values.  The following
gives an example of a simple namelist.

\begin{verbatim}
 &clmexp 
 CASEID         =  'test01'
 NSREST         =   0 
 NESTEP         =  -1 
 FSURDAT        =  '$CSMDATA/srfdata/cam/clms_64x128_T42_c020514.nc'
 FINIDAT        =  '$CSMDATA/inidata_2.1/cam/clmi_0000-09-01_64x128_T42_c021125.nc'
 FPFTCON        =  '$CSMDATA/pftdata/pft-physiology' 
 FRIVINP_RTM    =  '$CSMDATA/rtmdata/rdirc.05'
 OFFLINE_ATMDIR =  '$CSMDATA/NCEPDATA' 
 START_YMD      =  19971231
 DTIME          =  1800
 HIST_NHTFRQ(1) =  -24
 /
\end{verbatim}

\medskip \noindent 
{\bf CASEID} = 'test01' \\ Case identifier which distinguishes this
particular simulation from another. The string in {\bf CASEID} shows
up in the names of history, restart, and initial files, in the restart
pointer file name (see Example 2) and in the Mass Store pathname where
history and restart files are placed if the Mass Store is used.
In a branch run, the user must specify a new {\bf CASEID}.

\medskip \noindent 
{\bf NSREST} = 0 \\ Requests an initial run, as opposed to a restart
or a branch run. An initial run does not require the use of an initial
input datafile ({\bf FINIDAT}). If none is provided, the model uses
non spun-up initialization provided in the code (see
src/main/iniTimeVar.F90).

\medskip \noindent 
{\bf NESTEP} = -1 \\
Specifies the run's ending time to be at the end of day 1. Since {\bf
NESTEP} overrides any value given to {\bf NELAPSE}, {\bf NELAPSE}
has been omitted in this example.

\medskip \noindent 
{\bf FSURDAT} = '\$CSMDATA/srfdata/cam/clms\_64x128\_T42\_c020514.nc' \\ 
Specifies the name of the surface data input file. This T42 surface
dataset can be used both in cam and offline mode.  The model
resolution, (i.e. parameters {\bf lsmlon} and {\bf lsmlat}) must be
compatible with the resolution of {\bf FSURDAT}.  If the filename
appeared without a path specifying its exact location, the file would
be expected in the executable directory, defined by the environment
variable {\bf \$MODEL\_EXEDIR}.

\medskip \noindent 
{\bf FINIDAT} = '\$CSMDATA/inidata\_2.1/cam/clmi\_0000-09-01\_64x128\_T42\_c021125.nc' \\
Specifies the CLM2.1 initial file to be used instead of
model-specified initial values.  This results in a simulation that
starts from a spun-up state.

\medskip \noindent 
{\bf FPFTCON} = '\$CSMDATA/pftdata/pft-physiology' \\
Specifies a file with PFT (Plant Functional Type) information.  

\medskip \noindent 
{\bf FRIVINP\_RTM} = '\$CSMDATA/rtmdata/rdirc.05' \\
Specifies the input file required for the operation of RTM (the River
Transport Model of Branstetter et al). By default, RTM will operate at
half degree horizontal resolution and will be invoked every 3 hours,
where the RTM input fluxes are averaged over the 3 hour period.  If
the user wants the RTM scheme to be invoked every timestep, {\bf
RTM\_NSTEPS} should be set to 1. If the user wants the RTM scheme to
operate at a different frequency than once every 3 hours, {\bf
RTM\_NSTEPS} should be set to the desired value.  Use of RTM is
activated in the jobscript.csh with the C pre-processor (cpp) directive
\#define RTM in the header file {\bf preproc.h} (see \ref{subsubsec_cpp}).

\medskip \noindent 
{\bf OFFLINE\_ATMDIR} = '\$CSMDATA/NCEPDATA' \\
Specifies the location of the atmospheric driver data set. Such a data
set is required for the model to run in offline mode.

\medskip \noindent 
{\bf START\_YMD} = 19971231 \\ 
Specifies the base date of the simulation and must be compatible with
the atmospheric input data. For example, {\bf START\_YMD} = 19971231
will use the atmospheric input file 1997-12.nc. In a restart or branch
run, {\bf START\_YMD} need not be changed, as long as it refers to a
date earlier than the date of restart or branch.

\medskip \noindent 
{\bf DTIME} = 1800 \\
Specifies the simulation's timestep in seconds. In offline mode, the
model can handle a timestep of up to 3600 seconds.

\medskip \noindent 
{\bf HIST\_NHTFRQ(1)} = -24 \\
Primary history files and restart files will be produced in the
executable directory and will be written every 24 hours.

\subsection {Example 2: Restart run}

\noindent 
The following namelist generates a restart run starting from the last
file generated by Example 1.

\begin{verbatim}
 &clmexp
 CASEID         = 'test01' 
 NSREST         =  1 
 NELAPSE        = -1 
 FSURDAT        = '$CSMDATA/srfdata/cam/clms_64x128_T42_c020514.nc'
 FINIDAT        = '$CSMDATA/inidata_2.1/cam/clmi_0000-09-01_64x128_T42_c021125.nc'
 FPFTCON        = '$CSMDATA/pftdata/pft-physiology' 
 FRIVINP_RTM    = '$CSMDATA/rtmdata/rdirc.05'
 OFFLINE_ATMDIR = '$CSMDATA/NCEPDATA' 
 START_YMD      =  19971231 
 DTIME          =  1800 
 HIST_NHTFRQ(1) =  -24
 / 
\end{verbatim}

\medskip \noindent 
{\bf NSREST = 1} \\
Requests a restart run. A restart run finds the name of the
appropriate restart file automatically by reading the file, lnd.{\bf
CASEID}.rpointer.  In this example, the pointer file will be
lnd.test01.rpointer. Restart runs are meant to be 'seamless,'
producing the same output as runs which were not restarted.

\medskip \noindent 
{\bf NELAPSE} = -1 \\
Specifies the run's ending time to be one day after the point of
restart. This is equivalent to entering {\bf NESTEP} = -2 instead,
since the previous run stopped at the end of day 1.

\medskip \noindent 
All other namelist variables remain the same to ensure a 'seamless'
restart (for information, see example 1). Also, for a seamless
restart, the user should generally execute the code with the same
executable used in the initial run (ie, without compiling the code
again).

\subsection {Example 3: Branch run}

\noindent 
The following namelist generates a branch run starting from restart
files generated by Example 1.  The user may branch a run with the same
executable used in the initial run (i.e., without recompiling the
code) unless branching is used to test changes in the code (for
debugging or sensitivity purposes).

\begin{verbatim}
 &clmexp 
 CASEID         = 'branch_run' 
 NSREST         =  3 
 NELAPSE        = -1 
 FSURDAT        = '$CSMDATA/srfdata/cam/clms_64x128_T42_c020514.nc'
 FINIDAT        = '$CSMDATA/inidata_2.1/cam/clmi_0000-09-01_64x128_T42_c021125.nc'
 FPFTCON        = '$CSMDATA/pftdata/pft-physiology' 
 FRIVINP_RTM    = '$CSMDATA/rtmdata/rdirc.05'
 OFFLINE_ATMDIR = '$CSMDATA/NCEPDATA' 
 NREVSN         = 'test01.clm2.r.1998-01-01-00000' 
 HIST_FINC2     = 'TV:I'
 HIST_NHTFRQ    = -3,5 
 HIST_MFILT     =  2,3
 START_YMD      = 19971231 
 DTIME          = 1800 
 / 
\end{verbatim}

\medskip \noindent 
See Example 1 for explanations of namelist variables which remain
unchanged.

\medskip \noindent 
{\bf NSREST} = 3 \\
Requests a branch run.

\medskip \noindent 
{\bf NELAPSE} = -1 \\
Specifies the run's ending time to be one day after the
point of branching.

\medskip \noindent 
{\bf NREVSN} = 'test.clm2.r.1998-01-01-00000' \\
Supplies the name of the restart file which will initialize this run.
(Note this file can be produced by running example 1 above).

\medskip \noindent 
{\bf HIST\_FINCL2} = 'TV:I' \\
Add an auxiliary history file with the field ``TV'' that is output instantaneously.

\medskip \noindent 
{\bf HIST\_NHTFRQ} = -3,5 \\ 
Changes the frequency of primary history history writes to every 3
hours.  The write frequency of the auxiliary file is every 5 time
steps.  This is an example of a change which a user may wish to test
in a branch run.

\medskip \noindent 
{\bf HIST\_MFILT} =  2,3 \\
The primary history file will have 2 time samples on every tape. The
auxiliary history file will have 3 time samples on every tape.

\subsection {Example 4: Auxiliary history files}

\noindent 
This example covers the addition of an auxiliary history file, the
removal of a field from the primary history file and the change of
field type in a history file. A variety of other namelist options are
also illustrated.

\begin{verbatim}
 &CLMEXP    
 CASEID          = 'rtm_run' 
 NSREST          =  0 
 NESTEP          = -31 
 START_YMD       =  19980101 
 DTIME           =  1800 
 FSURDAT        =  '$CSMDATA/srfdata/cam/clms_64x128_T42_c020514.nc'
 FINIDAT        =  '$CSMDATA/inidata_2.1/cam/surface-data.120x060.013002.nc' 
 FRIVINP_RTM     = '$CSMDATA/rtmdata/rdirc.05' 
 OFFLINE_ATMDIR  = '$CSMDATA/NCEPDATA' 
 HIST_DOV2XY     = .true.,.false. 
 HIST_NHTFRQ     = -24,-12 
 HIST_MFILT      =  4,2 
 HIST_FINCL2     = 'TV','TG:I'
 HIST_FEXCL1     = 'TSNOW'
 MSS_IRT         = 365  
 WRTDIA          = .true. 
 / 
\end{verbatim}

\medskip \noindent 
For namelist variables which are repeated, refer to Examples 1, 2, and 3.

\medskip \noindent 
{\bf HIST\_DOV2XY} = .true.,.false. \\ 
History output will appear in gridded two-dimensional format
(rather than one-dimensional subgrid format) for the primary file and in
one-dimensional subgrid format for the auxiliary file.

\medskip \noindent 
{\bf HIST\_NHTFRQ} = -24,-12 \\ 
History output will be directed to the primary history file every 24
model hours and to the auxiliary history file every 12 hours.

\medskip \noindent 
{\bf HIST\_MFILT} = 4,2 \\ 
Each primary history file will contain 4 time slices of output, while
each auxiliary history file will contain 2 time slice of output.

\medskip \noindent 
{\bf HIST\_FINCL2} = 'TV','TG:I' \\ 
Specifies the two fields to be added to the auxiliary history output. 
The first field, 'TV', will have the default time averaging done,
whereas the second field, 'TG' will have instantaneous output.

\medskip \noindent 
{\bf HIST\_FEXCL1} = 'TSNOW' \\
The field 'TSNOW' will be excluded from the primary tape.

\medskip \noindent {\bf WRTDIA} = .true. \\ 
A global average of land surface air temperature as diagnostic will
appear in the standard output file of the simulation.

\medskip \noindent 
{\bf MSS\_IRT} = 365 \\ 
Output files will be archived to the NCAR Mass Storage System with a
retention time of 365 days.

\subsection {Example 5. Generation of regional grid surface dataset}

\medskip \noindent  
A regular grid surface dataset can be generated at run time for a
single gridcell or for gridcells comprising a regular regional or
global domain. To generate a surface dataset for a regional run, the
cpp tokens {\bf LSMLON} and {\bf LSMLAT} must be set to the desired
resolution (e.g., {\bf LSMLON}=1, {\bf LSMLAT}=1 for a single point
simulation) and the variables {\bf MKSRF\_OFFLINE\_EDGES}, {\bf
MKSRF\_OFFLINE\_EDGEN}, {\bf MKSRF\_OFFLINE\_EDGEE}, and {\bf
MKSRF\_OFFLINE\_EDGEW} and their values need to be added to the
namelist.  A surface dataset will be created with the name
surface-data.LSMLONxLSMLAT.nc (e.g., for a single point simulation the
file name will be surface-data.001x001.nc). The model can then be run
for the single point or for the regional domain by following Example 1
where {\bf FSURDAT} is set to the new surface dataset. 

In the following example, a regional grid is created over the
amazon. {\bf LSMLON} and {\bf LSMLAT} should be set to 15 and 11,
respectively.

\begin{verbatim}
 &clmexp     
 CASEID                 = 'create_regional_surfdat' 
 NSREST                 = 0 
 NESTEP                 = 2 
 START_YMD              = 19971231 
 DTIME                  = 1800 
 FSURDAT                = ' '   
 FRIVINP_RTM            = '$CSMDATA/rtmdata/rdirc.05' 
 FPFTCON                = '$CSMDATA/pftdata/pft-physiology' 
 OFFLINE_ATMDIR         = '$CSMDATA/NCEPDATA' 
 MKSRF_OFFLINE_FNAVYORO = '$CSMDATA/rawdata/mksrf_navyoro_20min.nc' 
 MKSRF_FVEGTYP          = '$CSMDATA/rawdata/mksrf_pft.nc' 
 MKSRF_FSOITEX          = '$CSMDATA/rawdata/mksrf_soitex.10level.nc' 
 MKSRF_FSOICOL          = '$CSMDATA/rawdata/mksrf_soicol_clm2.nc' 
 MKSRF_FLANWAT          = '$CSMDATA/rawdata/mksrf_lanwat.nc' 
 MKSRF_FGLACIER         = '$CSMDATA/rawdata/mksrf_glacier.nc' 
 MKSRF_FURBAN           = '$CSMDATA/rawdata/mksrf_urban.nc' 
 MKSRF_FLAI             = '$CSMDATA/rawdata/mksrf_lai.nc' 
 MKSRF_EDGEN            =  12 
 MKSRF_EDGES            = -21 
 MKSRF_EDGEE            = -36 
 MKSRF_EDGEW            = -81 
 /   
\end{verbatim}

\medskip \noindent 
{\bf FSURDAT} = ' ' \\ 
A surface dataset named surface-data.LSMLONXLSMLAT.nc will be created
in the model executable directory. LSMLON and LSMLAT are defined in
jobscript.csh (see \ref{subsubsec_cpp}).

\medskip \noindent
{\bf MKSRF\_OFFLINE\_FNAVYORO} = '\$CSMDATA/rawdata/mksrf\_navyoro\_20min.nc'\\ 
Points to the orography dataset used to derive the model's land mask
in offline mode.  The environment variable {\bf \$CSMDATA} is
explained in \ref{subsubsec_env_vars}.

\medskip \noindent 
{\bf MKSRF\_FVEGTYP}, {\bf MKSRF\_FSOITEX}, {\bf MKSRF\_FSOICOL}, {\bf
MKSRF\_FLANWAT}, {\bf MKSRF\_FGLACIER}, {\bf MKSRF\_FURBAN}, and {\bf
MKSRF\_FLAI} \\ 
Specify the raw (usually high resolution) input datasets used to
create the model surface dataset.

\medskip \noindent {\bf MKSRF\_OFFLINE\_EDGES}, {\bf MKSRF\_OFFLINE\_EDGEN}, {\bf
MKSRF\_OFFLINE\_EDGEE}, and {\bf MKSRF\_OFFLINE\_EDGEW} \\ 
Must be defined for the desired model regional domain.  The units are
degrees north for edges and edgen and degrees east for edgee and
edgew.

\subsection {Example 6. Generation of global gaussian surface dataset}

\medskip \noindent  
Only global surface datasets can be created on a non-regular grid,
such as a gaussian grid. To generate a surface dataset on a gaussian
grid, the cpp tokens {\bf LSMLON} and {\bf LSMLAT} must be set to the
desired resolution (e.g., {\bf LSMLON}=128, {\bf LSMLAT}=64 for a T42
grid, and {\bf MKSRF\_OFFLINE\_FGRID} must be set to the appropriate
dataset in {\bf \$CSMDATA/rawdata} specifying the model grid, land
mask and land fraction for the model grid. At T42 resolution, a
surface dataset, ``surface-data.128x064.nc'', will be created in the
model executable directory. This dataset may be renamed by the user to
be more self-explanatory.

\medskip \noindent 
The following namelist will result in the generation of a surface
dataset on a global gaussian grid.

\begin{verbatim}
 &clmexp 
 CASEID              = 'create_global_surfdat'  
 NSREST              =  0 
 NESTEP              =  2 
 START_YMD           = 19971231 
 DTIME               = 1800 
 FSURDAT             = ' '   
 FRIVINP_RTM         = '$CSMDATA/rtmdata/rdirc.05' 
 FPFTCON             = '$CSMDATA/pftdata/pft-physiology' 
 OFFLINE_ATMDIR      = '$CSMDATA/NCEPDATA' 
 MKSRF_OFFLINE_FGRID = '$CSMDATA/rawdata/T42_clm2_camfgrid_040802.nc' 
 MKSRF_FVEGTYP       = '$CSMDATA/rawdata/mksrf_pft.nc' 
 MKSRF_FSOITEX       = '$CSMDATA/rawdata/mksrf_soitex.10level.nc' 
 MKSRF_FSOICOL       = '$CSMDATA/rawdata/mksrf_soicol_clm2.nc' 
 MKSRF_FLANWAT       = '$CSMDATA/rawdata/mksrf_lanwat.nc' 
 MKSRF_FGLACIER      = '$CSMDATA/rawdata/mksrf_glacier.nc' 
 MKSRF_FURBAN        = '$CSMDATA/rawdata/mksrf_urban.nc' 
 MKSRF_FLAI          = '$CSMDATA/rawdata/mksrf_lai.nc' 
 / 
\end{verbatim}

\medskip \noindent 
{\bf FSURDAT} = ' ' \\ 
A surface dataset named surface-data.128x064.nc will be created at run
time in the model executable directory.

\medskip \noindent 
{\bf MKSRF\_OFFLINE\_FGRID} = '\$CSMDATA/rawdata/T42\_clm2\_camfgrid\_040802.nc'\\ 
Points to the dataset containing the model grid, land mask and
fractional land for the surface dataset.

\medskip \noindent 
{\bf MKSRF\_FVEGTYP}, {\bf MKSRF\_FSOITEX}, {\bf MKSRF\_FSOICOL}, {\bf
MKSRF\_FLANWAT}, {\bf MKSRF\_FGLACIER}, {\bf MKSRF\_FURBAN}, and {\bf
MKSRF\_FLAI} \\ 
Specifies the input datasets used to create the surface dataset.

